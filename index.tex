% Options for packages loaded elsewhere
% Options for packages loaded elsewhere
\PassOptionsToPackage{unicode}{hyperref}
\PassOptionsToPackage{hyphens}{url}
%
\documentclass[
  ignorenonframetext,
  british,
]{beamer}
\newif\ifbibliography
\usepackage{pgfpages}
\setbeamertemplate{caption}[numbered]
\setbeamertemplate{caption label separator}{: }
\setbeamercolor{caption name}{fg=normal text.fg}
\beamertemplatenavigationsymbolsempty
% remove section numbering
\setbeamertemplate{part page}{
  \centering
  \begin{beamercolorbox}[sep=16pt,center]{part title}
    \usebeamerfont{part title}\insertpart\par
  \end{beamercolorbox}
}
\setbeamertemplate{section page}{
  \centering
  \begin{beamercolorbox}[sep=12pt,center]{section title}
    \usebeamerfont{section title}\insertsection\par
  \end{beamercolorbox}
}
\setbeamertemplate{subsection page}{
  \centering
  \begin{beamercolorbox}[sep=8pt,center]{subsection title}
    \usebeamerfont{subsection title}\insertsubsection\par
  \end{beamercolorbox}
}
% Prevent slide breaks in the middle of a paragraph
\widowpenalties 1 10000
\raggedbottom
\AtBeginPart{
  \frame{\partpage}
}
\AtBeginSection{
  \ifbibliography
  \else
    \frame{\sectionpage}
  \fi
}
\AtBeginSubsection{
  \frame{\subsectionpage}
}
\usepackage{iftex}
\ifPDFTeX
  \usepackage[T1]{fontenc}
  \usepackage[utf8]{inputenc}
  \usepackage{textcomp} % provide euro and other symbols
\else % if luatex or xetex
  \usepackage{unicode-math} % this also loads fontspec
  \defaultfontfeatures{Scale=MatchLowercase}
  \defaultfontfeatures[\rmfamily]{Ligatures=TeX,Scale=1}
\fi
\usepackage{lmodern}

\ifPDFTeX\else
  % xetex/luatex font selection
\fi
% Use upquote if available, for straight quotes in verbatim environments
\IfFileExists{upquote.sty}{\usepackage{upquote}}{}
\IfFileExists{microtype.sty}{% use microtype if available
  \usepackage[]{microtype}
  \UseMicrotypeSet[protrusion]{basicmath} % disable protrusion for tt fonts
}{}
\makeatletter
\@ifundefined{KOMAClassName}{% if non-KOMA class
  \IfFileExists{parskip.sty}{%
    \usepackage{parskip}
  }{% else
    \setlength{\parindent}{0pt}
    \setlength{\parskip}{6pt plus 2pt minus 1pt}}
}{% if KOMA class
  \KOMAoptions{parskip=half}}
\makeatother


\usepackage{longtable,booktabs,array}
\newcounter{none} % for unnumbered tables
\usepackage{calc} % for calculating minipage widths
\usepackage{caption}
% Make caption package work with longtable
\makeatletter
\def\fnum@table{\tablename~\thetable}
\makeatother
\usepackage{graphicx}
\makeatletter
\newsavebox\pandoc@box
\newcommand*\pandocbounded[1]{% scales image to fit in text height/width
  \sbox\pandoc@box{#1}%
  \Gscale@div\@tempa{\textheight}{\dimexpr\ht\pandoc@box+\dp\pandoc@box\relax}%
  \Gscale@div\@tempb{\linewidth}{\wd\pandoc@box}%
  \ifdim\@tempb\p@<\@tempa\p@\let\@tempa\@tempb\fi% select the smaller of both
  \ifdim\@tempa\p@<\p@\scalebox{\@tempa}{\usebox\pandoc@box}%
  \else\usebox{\pandoc@box}%
  \fi%
}
% Set default figure placement to htbp
\def\fps@figure{htbp}
\makeatother



\ifLuaTeX
\usepackage[bidi=basic,shorthands=off]{babel}
\else
\usepackage[bidi=default,shorthands=off]{babel}
\fi
\ifLuaTeX
  \usepackage{selnolig} % disable illegal ligatures
\fi


\setlength{\emergencystretch}{3em} % prevent overfull lines

\providecommand{\tightlist}{%
  \setlength{\itemsep}{0pt}\setlength{\parskip}{0pt}}



 


\usepackage{standalone}
\usepackage{tikz}
\usepackage{pgfplots}
\usetikzlibrary{decorations.pathreplacing}
\usetikzlibrary{arrows.meta}
\pgfplotsset{compat=1.18}
\makeatletter
\@ifpackageloaded{tcolorbox}{}{\usepackage[skins,breakable]{tcolorbox}}
\@ifpackageloaded{fontawesome5}{}{\usepackage{fontawesome5}}
\definecolor{quarto-callout-color}{HTML}{909090}
\definecolor{quarto-callout-note-color}{HTML}{0758E5}
\definecolor{quarto-callout-important-color}{HTML}{CC1914}
\definecolor{quarto-callout-warning-color}{HTML}{EB9113}
\definecolor{quarto-callout-tip-color}{HTML}{00A047}
\definecolor{quarto-callout-caution-color}{HTML}{FC5300}
\definecolor{quarto-callout-color-frame}{HTML}{acacac}
\definecolor{quarto-callout-note-color-frame}{HTML}{4582ec}
\definecolor{quarto-callout-important-color-frame}{HTML}{d9534f}
\definecolor{quarto-callout-warning-color-frame}{HTML}{f0ad4e}
\definecolor{quarto-callout-tip-color-frame}{HTML}{02b875}
\definecolor{quarto-callout-caution-color-frame}{HTML}{fd7e14}
\makeatother
\makeatletter
\@ifpackageloaded{caption}{}{\usepackage{caption}}
\AtBeginDocument{%
\ifdefined\contentsname
  \renewcommand*\contentsname{Table of contents}
\else
  \newcommand\contentsname{Table of contents}
\fi
\ifdefined\listfigurename
  \renewcommand*\listfigurename{List of Figures}
\else
  \newcommand\listfigurename{List of Figures}
\fi
\ifdefined\listtablename
  \renewcommand*\listtablename{List of Tables}
\else
  \newcommand\listtablename{List of Tables}
\fi
\ifdefined\figurename
  \renewcommand*\figurename{Figure}
\else
  \newcommand\figurename{Figure}
\fi
\ifdefined\tablename
  \renewcommand*\tablename{Table}
\else
  \newcommand\tablename{Table}
\fi
}
\@ifpackageloaded{float}{}{\usepackage{float}}
\floatstyle{ruled}
\@ifundefined{c@chapter}{\newfloat{codelisting}{h}{lop}}{\newfloat{codelisting}{h}{lop}[chapter]}
\floatname{codelisting}{Listing}
\newcommand*\listoflistings{\listof{codelisting}{List of Listings}}
\makeatother
\makeatletter
\makeatother
\makeatletter
\@ifpackageloaded{caption}{}{\usepackage{caption}}
\@ifpackageloaded{subcaption}{}{\usepackage{subcaption}}
\makeatother

\usepackage{bookmark}
\IfFileExists{xurl.sty}{\usepackage{xurl}}{} % add URL line breaks if available
\urlstyle{same}
\hypersetup{
  pdfauthor={Gerhard Riener},
  pdflang={en-GB},
  hidelinks,
  pdfcreator={LaTeX via pandoc}}


\author{Gerhard Riener}
\date{}

\begin{document}


\begin{frame}{Welcome}
\protect\phantomsection\label{welcome}
This is the course website for \textbf{EC5230 Industrial Organisation}
at the University of St Andrews.

Industrial Organisation studies the structure of firms and markets,
strategic interactions between firms, and the effects of market power on
economic outcomes. This course covers both theoretical foundations and
practical applications, examining how firms behave in imperfectly
competitive markets.
\end{frame}

\begin{frame}{Course Overview}
\protect\phantomsection\label{course-overview}
\textbf{Module Credits:} 15 (Semester 2, Optional)

\begin{block}{Topics Covered}
\protect\phantomsection\label{topics-covered}
This module explores key concepts in industrial economics across twelve
weeks:

\begin{itemize}
\tightlist
\item
  \textbf{Competition and Oligopoly Coordination} (Week 1)
\item
  \textbf{Product Differentiation} (Week 2)
\item
  \textbf{Economics of Innovation} (Week 3)
\item
  \textbf{Economics of Patents and IPRs} (Week 4)
\item
  \textbf{Multi-stage Games} (Week 5)
\item
  \textbf{Cooperative R\&D} (Week 7)
\item
  \textbf{Bundling} (Week 8)
\item
  \textbf{Advertising} (Week 9)
\item
  \textbf{Mergers} (Week 11)
\item
  \textbf{Sustainable Industrialisation} (Week 12)
\end{itemize}
\end{block}

\begin{block}{Intended Learning Outcomes}
\protect\phantomsection\label{intended-learning-outcomes}
By the end of the module, you will be able to:

\begin{itemize}
\tightlist
\item
  Understand how Industrial Organisation deepens the analysis of firm
  behaviour in imperfectly competitive markets
\item
  Select and apply appropriate models to analyse various issues in
  industrial economics
\item
  Develop skills in analysing economic behaviour through game theory
\item
  Identify and critically evaluate relevant research papers and
  real-world examples
\end{itemize}
\end{block}

\begin{block}{Sustainability in the Curriculum}
\protect\phantomsection\label{sustainability-in-the-curriculum}
This module contributes to UN Sustainable Development Goals:

\begin{itemize}
\tightlist
\item
  \textbf{Goal 9: Industry, Innovation, and Infrastructure} ---
  Examining how patents and IPRs impact innovation disclosure and
  cooperative R\&D
\item
  \textbf{Goal 12: Responsible Consumption and Production} --- Analysing
  network effects in technology adoption and sustainable
  industrialisation
\end{itemize}
\end{block}
\end{frame}

\begin{frame}{Course Materials}
\protect\phantomsection\label{course-materials}
\begin{tcolorbox}[enhanced jigsaw, coltitle=black, bottomrule=.15mm, titlerule=0mm, left=2mm, rightrule=.15mm, toprule=.15mm, leftrule=.75mm, title=\textcolor{quarto-callout-note-color}{\faInfo}\hspace{0.5em}{Navigation}, colback=white, colframe=quarto-callout-note-color-frame, breakable, toptitle=1mm, bottomtitle=1mm, arc=.35mm, opacitybacktitle=0.6, colbacktitle=quarto-callout-note-color!10!white, opacityback=0]

Use the links below to access lectures, exercises, and interactive
applications.

\end{tcolorbox}

\begin{block}{\href{lecture-slides/}{Lecture Slides}}
\protect\phantomsection\label{lecture-slides}
Comprehensive lecture notes covering key topics in industrial
organisation:

\begin{itemize}
\tightlist
\item
  \href{lecture-slides/lecture-1-oligopoly.html}{Lecture 1: Competition
  \& Oligopoly Coordination}
\item
  \href{lecture-slides/lecture-2-product-differentiation.html}{Lecture
  2: Product Differentiation}
\item
  \href{lecture-slides/lecture-3-innovation.html}{Lecture 3: Economics
  of Innovation}
\item
  \href{lecture-slides/lecture-4-patents.html}{Lecture 4: Patents and
  Intellectual Property Rights}
\item
  \href{lecture-slides/lecture-5-multistage.html}{Lecture 5: Multi-stage
  Games and Strategic Commitment}
\item
  \href{lecture-slides/lecture-7-cooperative-rd.html}{Lecture 7:
  Cooperative R\&D}
\item
  \href{lecture-slides/lecture-8-bundling.html}{Lecture 8: Bundling and
  Tying}
\item
  \href{lecture-slides/lecture-9-advertising.html}{Lecture 9:
  Advertising and Brand Competition}
\item
  \href{lecture-slides/lecture-11-mergers.html}{Lecture 11: Mergers and
  Acquisitions}
\item
  \href{lecture-slides/lecture-12-sustainability.html}{Lecture 12:
  Sustainable Industrialisation}
\end{itemize}
\end{block}

\begin{block}{\href{exercises/}{Exercises}}
\protect\phantomsection\label{exercises}
Problem sets with and without solutions to help you practice the
concepts covered in lectures. Tutorials focus on live problem-solving
(numerical and discussion/interpretation questions).
\end{block}

\begin{block}{\href{apps/}{Interactive Apps}}
\protect\phantomsection\label{interactive-apps}
Interactive visualisations to explore key economic models:

\begin{itemize}
\tightlist
\item
  Cournot duopoly best responses
\item
  Monopolist profit maximisation
\item
  Profit possibility frontiers
\item
  Salop circular city model
\end{itemize}
\end{block}
\end{frame}

\begin{frame}{Textbooks and Readings}
\protect\phantomsection\label{textbooks-and-readings}
There is no single prescribed textbook. The course draws on chapters
from:

\begin{itemize}
\tightlist
\item
  \textbf{Belleflamme, P. \& Peitz, M. (2015).} \emph{Industrial
  Organisation: Markets and Strategies} (2nd ed.). Cambridge University
  Press.
\item
  \textbf{Shy, O. (1996).} \emph{Industrial Organisation: Theory and
  Applications}. MIT Press.
\item
  \textbf{Scotchmer, S. (2004).} \emph{Innovation and Incentives}.
  Princeton University Press.
\item
  \textbf{Church, J. \& Ware, R. (2000).} \emph{Industrial Organization:
  A Strategic Approach}. Irwin McGraw-Hill.
\end{itemize}

Journal articles are available through the online reading list and cited
at the end of each lecture.
\end{frame}

\begin{frame}{Assessment}
\protect\phantomsection\label{assessment}
{\def\LTcaptype{none} % do not increment counter
\begin{longtable}[]{@{}lll@{}}
\toprule\noalign{}
Component & Weight & Timing \\
\midrule\noalign{}
\endhead
Class Test & 20\% & Week 6 \\
Policy Recommendation & 5\% & Week 11 \\
Final Examination & 75\% & May \\
\bottomrule\noalign{}
\end{longtable}
}

\begin{itemize}
\tightlist
\item
  \textbf{Class Test:} 50-minute in-person test covering Weeks 1--5,
  including a problem-solving question and a discussion-based question
  (Week 6, during lecture slot)
\item
  \textbf{Policy Recommendation:} Written report (max 1,000 words)
  analysing a real-world industrial organisation issue of your choice,
  submitted via MMS by \textbf{15 April 2026, 17:00}
\item
  \textbf{Final Examination:} Three-hour in-person exam covering all
  module content (Weeks 1--5, 7--9, 11--12)
\end{itemize}
\end{frame}

\begin{frame}{Prerequisites}
\protect\phantomsection\label{prerequisites}
\begin{itemize}
\tightlist
\item
  Good understanding of intermediate microeconomics (market structures,
  competition, welfare analysis, game theory)
\item
  Basic algebra and calculus (optimisation)
\item
  No econometric or statistical knowledge required
\end{itemize}
\end{frame}

\begin{frame}{Course Structure}
\protect\phantomsection\label{course-structure}
\begin{itemize}
\tightlist
\item
  \textbf{Lectures:} 10 × 2 hours (Weeks 1--5, 7--9, 11--12; no lecture
  during Spring Break Week 6 or Reading Week 10)

  \begin{itemize}
  \tightlist
  \item
    \textbf{Week 1:} Wednesday 28 January 2026, 09:00--11:00
  \item
    \textbf{Week 2:} Wednesday 04 February 2026, 09:00--11:00
  \item
    \textbf{Week 3:} Wednesday 11 February 2026, 09:00--11:00
  \item
    \textbf{Week 4:} Wednesday 18 February 2026, 09:00--11:00
  \item
    \textbf{Week 5:} Wednesday 25 February 2026, 09:00--11:00
  \item
    \textbf{Spring Break:} 02--08 March 2026
  \item
    \textbf{Week 6:} Class Test (Wednesday 11 March 2026, 09:00--10:30)
  \item
    \textbf{Week 7:} Wednesday 18 March 2026, 09:00--11:00
  \item
    \textbf{Week 8:} no lecture scheduled
  \item
    \textbf{Week 9:} no lecture scheduled
  \item
    \textbf{Week 10:} Independent Learning Week (06--12 April 2026)
  \item
    \textbf{Week 11:} Wednesday 15 April 2026, 08:00--10:00
  \item
    \textbf{Week 12:} Wednesday 22 April 2026, 08:00--10:00
  \end{itemize}
\item
  \textbf{Tutorials:} 5 × 1 hour (Weeks 3, 5, 7, 11, 12)

  \begin{itemize}
  \tightlist
  \item
    \textbf{Week 3:} Monday 09 February 2026, 14:00--15:00
  \item
    \textbf{Week 5:} Monday 23 February 2026, 14:00--15:00
  \item
    \textbf{Week 7:} Monday 16 March 2026, 14:00--15:00
  \item
    \textbf{Week 11:} Monday 13 April 2026, 13:00--14:00
  \item
    \textbf{Week 12:} Monday 20 April 2026, 13:00--14:00
  \end{itemize}
\end{itemize}
\end{frame}

\begin{frame}{Contact}
\protect\phantomsection\label{contact}
\textbf{Instructor:} Prof Gerhard Riener

\begin{itemize}
\tightlist
\item
  \textbf{Email:} gr97@st-andrews.ac.uk
\item
  \textbf{Office:} G4C Castlecliffe
\item
  \textbf{Office Hours:} Wednesday 12--1pm, 2--3pm in lecture weeks
\end{itemize}

For questions about the course, please feel free to reach out via email
or visit during office hours. Book here up to 12 hours before:
\href{https://outlook.office.com/bookwithme/user/0f38b6e3379248f0a4f92f3e56bf9a51@st-andrews.ac.uk/meetingtype/EcyHhzFPYEGq-SsSBEwTiw2?anonymous&ep=mlink}{Book}
\end{frame}

\begin{frame}
\emph{Course materials developed at the University of St Andrews.
Interactive apps based on concepts by Flavio Toxvaerd (University of
Cambridge).}
\end{frame}




\end{document}
