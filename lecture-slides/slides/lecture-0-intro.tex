% Options for packages loaded elsewhere
% Options for packages loaded elsewhere
\PassOptionsToPackage{unicode}{hyperref}
\PassOptionsToPackage{hyphens}{url}
%
\documentclass[
  ignorenonframetext,
  british,
]{beamer}
\newif\ifbibliography
\usepackage{pgfpages}
\setbeamertemplate{caption}[numbered]
\setbeamertemplate{caption label separator}{: }
\setbeamercolor{caption name}{fg=normal text.fg}
\beamertemplatenavigationsymbolsempty
% remove section numbering
\setbeamertemplate{part page}{
  \centering
  \begin{beamercolorbox}[sep=16pt,center]{part title}
    \usebeamerfont{part title}\insertpart\par
  \end{beamercolorbox}
}
\setbeamertemplate{section page}{
  \centering
  \begin{beamercolorbox}[sep=12pt,center]{section title}
    \usebeamerfont{section title}\insertsection\par
  \end{beamercolorbox}
}
\setbeamertemplate{subsection page}{
  \centering
  \begin{beamercolorbox}[sep=8pt,center]{subsection title}
    \usebeamerfont{subsection title}\insertsubsection\par
  \end{beamercolorbox}
}
% Prevent slide breaks in the middle of a paragraph
\widowpenalties 1 10000
\raggedbottom
\AtBeginPart{
  \frame{\partpage}
}
\AtBeginSection{
  \ifbibliography
  \else
    \frame{\sectionpage}
  \fi
}
\AtBeginSubsection{
  \frame{\subsectionpage}
}
\usepackage{iftex}
\ifPDFTeX
  \usepackage[T1]{fontenc}
  \usepackage[utf8]{inputenc}
  \usepackage{textcomp} % provide euro and other symbols
\else % if luatex or xetex
  \usepackage{unicode-math} % this also loads fontspec
  \defaultfontfeatures{Scale=MatchLowercase}
  \defaultfontfeatures[\rmfamily]{Ligatures=TeX,Scale=1}
\fi
\usepackage{lmodern}

\ifPDFTeX\else
  % xetex/luatex font selection
\fi
% Use upquote if available, for straight quotes in verbatim environments
\IfFileExists{upquote.sty}{\usepackage{upquote}}{}
\IfFileExists{microtype.sty}{% use microtype if available
  \usepackage[]{microtype}
  \UseMicrotypeSet[protrusion]{basicmath} % disable protrusion for tt fonts
}{}
\makeatletter
\@ifundefined{KOMAClassName}{% if non-KOMA class
  \IfFileExists{parskip.sty}{%
    \usepackage{parskip}
  }{% else
    \setlength{\parindent}{0pt}
    \setlength{\parskip}{6pt plus 2pt minus 1pt}}
}{% if KOMA class
  \KOMAoptions{parskip=half}}
\makeatother


\usepackage{longtable,booktabs,array}
\newcounter{none} % for unnumbered tables
\usepackage{calc} % for calculating minipage widths
\usepackage{caption}
% Make caption package work with longtable
\makeatletter
\def\fnum@table{\tablename~\thetable}
\makeatother
\usepackage{graphicx}
\makeatletter
\newsavebox\pandoc@box
\newcommand*\pandocbounded[1]{% scales image to fit in text height/width
  \sbox\pandoc@box{#1}%
  \Gscale@div\@tempa{\textheight}{\dimexpr\ht\pandoc@box+\dp\pandoc@box\relax}%
  \Gscale@div\@tempb{\linewidth}{\wd\pandoc@box}%
  \ifdim\@tempb\p@<\@tempa\p@\let\@tempa\@tempb\fi% select the smaller of both
  \ifdim\@tempa\p@<\p@\scalebox{\@tempa}{\usebox\pandoc@box}%
  \else\usebox{\pandoc@box}%
  \fi%
}
% Set default figure placement to htbp
\def\fps@figure{htbp}
\makeatother



\ifLuaTeX
\usepackage[bidi=basic,shorthands=off]{babel}
\else
\usepackage[bidi=default,shorthands=off]{babel}
\fi
\ifLuaTeX
  \usepackage{selnolig} % disable illegal ligatures
\fi


\setlength{\emergencystretch}{3em} % prevent overfull lines

\providecommand{\tightlist}{%
  \setlength{\itemsep}{0pt}\setlength{\parskip}{0pt}}



 


\usepackage{standalone}
\usepackage{tikz}
\usepackage{pgfplots}
\usetikzlibrary{decorations.pathreplacing}
\usetikzlibrary{arrows.meta}
\pgfplotsset{compat=1.18}
\makeatletter
\@ifpackageloaded{tcolorbox}{}{\usepackage[skins,breakable]{tcolorbox}}
\@ifpackageloaded{fontawesome5}{}{\usepackage{fontawesome5}}
\definecolor{quarto-callout-color}{HTML}{909090}
\definecolor{quarto-callout-note-color}{HTML}{0758E5}
\definecolor{quarto-callout-important-color}{HTML}{CC1914}
\definecolor{quarto-callout-warning-color}{HTML}{EB9113}
\definecolor{quarto-callout-tip-color}{HTML}{00A047}
\definecolor{quarto-callout-caution-color}{HTML}{FC5300}
\definecolor{quarto-callout-color-frame}{HTML}{acacac}
\definecolor{quarto-callout-note-color-frame}{HTML}{4582ec}
\definecolor{quarto-callout-important-color-frame}{HTML}{d9534f}
\definecolor{quarto-callout-warning-color-frame}{HTML}{f0ad4e}
\definecolor{quarto-callout-tip-color-frame}{HTML}{02b875}
\definecolor{quarto-callout-caution-color-frame}{HTML}{fd7e14}
\makeatother
\makeatletter
\@ifpackageloaded{caption}{}{\usepackage{caption}}
\AtBeginDocument{%
\ifdefined\contentsname
  \renewcommand*\contentsname{Table of contents}
\else
  \newcommand\contentsname{Table of contents}
\fi
\ifdefined\listfigurename
  \renewcommand*\listfigurename{List of Figures}
\else
  \newcommand\listfigurename{List of Figures}
\fi
\ifdefined\listtablename
  \renewcommand*\listtablename{List of Tables}
\else
  \newcommand\listtablename{List of Tables}
\fi
\ifdefined\figurename
  \renewcommand*\figurename{Figure}
\else
  \newcommand\figurename{Figure}
\fi
\ifdefined\tablename
  \renewcommand*\tablename{Table}
\else
  \newcommand\tablename{Table}
\fi
}
\@ifpackageloaded{float}{}{\usepackage{float}}
\floatstyle{ruled}
\@ifundefined{c@chapter}{\newfloat{codelisting}{h}{lop}}{\newfloat{codelisting}{h}{lop}[chapter]}
\floatname{codelisting}{Listing}
\newcommand*\listoflistings{\listof{codelisting}{List of Listings}}
\makeatother
\makeatletter
\makeatother
\makeatletter
\@ifpackageloaded{caption}{}{\usepackage{caption}}
\@ifpackageloaded{subcaption}{}{\usepackage{subcaption}}
\makeatother

\usepackage{bookmark}
\IfFileExists{xurl.sty}{\usepackage{xurl}}{} % add URL line breaks if available
\urlstyle{same}
\hypersetup{
  pdftitle={EC5230 -- Industrial Organisation},
  pdfauthor={Gerhard Riener},
  pdflang={en-GB},
  hidelinks,
  pdfcreator={LaTeX via pandoc}}


\title{\texorpdfstring{EC5230 -- Industrial
Organisation}{EC5230 -- Industrial Organisation}}
\subtitle{\texorpdfstring{Lecture 0 --
Introduction}{Lecture 0 -- Introduction}}
\author{Gerhard Riener}
\date{}

\begin{document}
\frame{\titlepage}


\begin{frame}{Course Overview}
\protect\phantomsection\label{course-overview}
\textbf{EC5230 Industrial Organisation}

\begin{itemize}
\tightlist
\item
  \textbf{Credits:} 15 (Semester 2, Optional)
\item
  \textbf{Instructor:} Prof Gerhard Riener
\item
  \textbf{Email:} gr97@st-andrews.ac.uk
\item
  \textbf{Office:} G14 Castlecliffe
\item
  \textbf{Office Hours:} Wednesday 12--1pm, 2--3pm
\end{itemize}
\end{frame}

\begin{frame}{What is Industrial Organisation?}
\protect\phantomsection\label{what-is-industrial-organisation}
Industrial Organisation studies:

\begin{itemize}
\tightlist
\item
  Structure of firms and markets
\item
  Strategic interactions between firms
\item
  Effects of market power on economic outcomes
\item
  How firms behave in imperfectly competitive markets
\end{itemize}
\end{frame}

\begin{frame}{Prerequisites \& Expectations}
\protect\phantomsection\label{prerequisites-expectations}
Prerequisites

\begin{itemize}
\tightlist
\item
  You should remind yourself of basic algebra, calculus
  (optimisation).\\
\item
  A good understanding of intermediate microeconomics concepts\\
  (market structures, competition, welfare analysis, game\\
  theory) is essential.\\
\item
  Econometric or statistical knowledge is not required.
\end{itemize}

Expectations

\begin{itemize}
\tightlist
\item
  Attend the lectures and tutorials -- ask and answer questions.\\
\item
  Go through the required readings carefully.\\
\item
  Derive (again) the results from lectures, tutorials, readings\\
  -- and reflect on them.\\
\item
  Attempt problem set questions before attending the tutorials.
\end{itemize}
\end{frame}

\begin{frame}{Learning Outcomes}
\protect\phantomsection\label{learning-outcomes}
By the end of this module, you will be able to:

\begin{itemize}
\tightlist
\item
  Understand how Industrial Organisation deepens the analysis of firm
  behaviour in imperfectly competitive markets
\item
  Select and apply appropriate models to analyse various issues in
  industrial economics
\item
  Develop skills in analysing economic behaviour through game theory
\item
  Identify and critically evaluate relevant research papers and
  real-world examples
\end{itemize}
\end{frame}

\begin{frame}{Course Structure}
\protect\phantomsection\label{course-structure}
\begin{itemize}
\tightlist
\item
  \textbf{Lectures:} 10 × 2 hours (Weeks 1--5, 7--9, 11--12; no lecture
  during Spring Break)
\item
  \textbf{Tutorials:} 5 × 1 hour (Weeks 3, 5, 7, 9, 12)
\item
  \textbf{Independent Learning Week:} Week 10
\end{itemize}

\begin{tcolorbox}[enhanced jigsaw, coltitle=black, arc=.35mm, toprule=.15mm, leftrule=.75mm, colbacktitle=quarto-callout-important-color!10!white, opacityback=0, bottomtitle=1mm, opacitybacktitle=0.6, breakable, colframe=quarto-callout-important-color-frame, bottomrule=.15mm, colback=white, titlerule=0mm, title=\textcolor{quarto-callout-important-color}{\faExclamation}\hspace{0.5em}{Important}, rightrule=.15mm, left=2mm, toptitle=1mm]

\textbf{Attendance during tutorials will be taken electronically}

\end{tcolorbox}
\end{frame}

\begin{frame}{Assessment and Examination}
\protect\phantomsection\label{assessment-and-examination}
{\def\LTcaptype{none} % do not increment counter
\begin{longtable}[]{@{}
  >{\raggedright\arraybackslash}p{(\linewidth - 6\tabcolsep) * \real{0.3235}}
  >{\raggedright\arraybackslash}p{(\linewidth - 6\tabcolsep) * \real{0.2353}}
  >{\raggedright\arraybackslash}p{(\linewidth - 6\tabcolsep) * \real{0.1765}}
  >{\raggedright\arraybackslash}p{(\linewidth - 6\tabcolsep) * \real{0.2647}}@{}}
\toprule\noalign{}
\begin{minipage}[b]{\linewidth}\raggedright
Component
\end{minipage} & \begin{minipage}[b]{\linewidth}\raggedright
Weight
\end{minipage} & \begin{minipage}[b]{\linewidth}\raggedright
When
\end{minipage} & \begin{minipage}[b]{\linewidth}\raggedright
Details
\end{minipage} \\
\midrule\noalign{}
\endhead
\textbf{Class Test} & 20\% & Week 6 & 50 minutes, in-person.
Problem-solving and discussion-based questions on Weeks 1--5 \\
\textbf{Policy Recommendation} & 5\% & 15 Apr 2026, 17:00 & Report (max
1,000 words) on a real-world IO issue. Submit via MMS \\
\textbf{Final Examination} & 75\% & May & Three hours, in-person. All
module content (Weeks 1--5, 7--9, 11--12). Guidelines provided closer to
exam \\
\bottomrule\noalign{}
\end{longtable}
}
\end{frame}

\begin{frame}{Prerequisites}
\protect\phantomsection\label{prerequisites}
\begin{itemize}
\tightlist
\item
  Good understanding of intermediate microeconomics

  \begin{itemize}
  \tightlist
  \item
    Market structures, competition, welfare analysis
  \item
    Game theory
  \end{itemize}
\item
  Basic algebra and calculus (optimisation)
\item
  \textbf{No econometric or statistical knowledge required}
\end{itemize}
\end{frame}

\begin{frame}{Textbooks and Readings}
\protect\phantomsection\label{textbooks-and-readings}
No single prescribed textbook. The course draws on:

\begin{itemize}
\tightlist
\item
  \textbf{Belleflamme, P. \& Peitz, M. (2015).} \emph{Industrial
  Organisation: Markets and Strategies} (2nd ed.)
\item
  \textbf{Shy, O. (1996).} \emph{Industrial Organisation: Theory and
  Applications}
\item
  \textbf{Scotchmer, S. (2004).} \emph{Innovation and Incentives}
\item
  \textbf{Church, J. \& Ware, R. (2000).} \emph{Industrial Organization:
  A Strategic Approach}
\end{itemize}

Journal articles available through online reading list.
\end{frame}

\begin{frame}{Course Website}
\protect\phantomsection\label{course-website}
\textbf{All materials available online:}

\begin{itemize}
\tightlist
\item
  Lecture slides
\item
  Exercise sets (with/without solutions)
\item
  Interactive apps for key economic models
\item
  Reading lists
\end{itemize}

Use the navigation bar to access different sections.
\end{frame}

\begin{frame}{Interactive Apps}
\protect\phantomsection\label{interactive-apps}
Explore key economic models through visualisations:

\begin{itemize}
\tightlist
\item
  Cournot duopoly best responses
\item
  Monopolist profit maximisation
\item
  Profit possibility frontiers
\item
  Salop circular city model
\end{itemize}
\end{frame}

\begin{frame}{Administrative Procedures}
\protect\phantomsection\label{administrative-procedures}
For the following, we follow \textbf{standard Department of Economics /
Business School / University of St Andrews procedures}:

\begin{itemize}
\tightlist
\item
  \textbf{Late submission policy} --- Policy briefs submitted after the
  deadline incur penalties per university regulations
\item
  \textbf{Absence \& extenuating circumstances} --- Contact the School
  immediately; documented evidence required (illness, personal crisis,
  etc.)
\item
  \textbf{Feedback on assessment} --- Results released within standard
  university timescales; feedback available during office hours or by
  email
\item
  \textbf{Communication \& support} --- Email gr97@st-andrews.ac.uk for
  queries; response within 2 working days. Office hours: Wednesday
  12--1pm, 2--3pm
\item
  \textbf{Exam accommodation} --- Register with Student Services well in
  advance for access arrangements, extensions, or reasonable adjustments
\end{itemize}

Full details available on the \textbf{Business School and University
websites}. https://www.st-andrews.ac.uk/pgstudents/
\end{frame}

\begin{frame}{Weekly Schedule}
\protect\phantomsection\label{weekly-schedule}
\textbf{Week 1} Competition and oligopoly coordination\\
\textbf{Week 2} Product differentiation\\
\textbf{Week 3} Economics of innovation\\
\textbf{Week 4} Economics of patents and IPRs\\
\textbf{Week 5} Multi-stage games\\
\textbf{Spring Vacation}\\
\textbf{Week 6} CLASS TEST (20\% Final Grade)\\
\textbf{Week 7} Cooperative R\&D\\
\textbf{Week 8} Bundling\\
\textbf{Week 9} Advertising\\
\textbf{Week 10} Independent Learning Week\\
\textbf{Week 11} Mergers\\
\textbf{Week 12} Sustainable industrialisation
\end{frame}




\end{document}
