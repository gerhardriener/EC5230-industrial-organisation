% Options for packages loaded elsewhere
% Options for packages loaded elsewhere
\PassOptionsToPackage{unicode}{hyperref}
\PassOptionsToPackage{hyphens}{url}
%
\documentclass[
  ignorenonframetext,
  british,
]{beamer}
\newif\ifbibliography
\usepackage{pgfpages}
\setbeamertemplate{caption}[numbered]
\setbeamertemplate{caption label separator}{: }
\setbeamercolor{caption name}{fg=normal text.fg}
\beamertemplatenavigationsymbolsempty
% remove section numbering
\setbeamertemplate{part page}{
  \centering
  \begin{beamercolorbox}[sep=16pt,center]{part title}
    \usebeamerfont{part title}\insertpart\par
  \end{beamercolorbox}
}
\setbeamertemplate{section page}{
  \centering
  \begin{beamercolorbox}[sep=12pt,center]{section title}
    \usebeamerfont{section title}\insertsection\par
  \end{beamercolorbox}
}
\setbeamertemplate{subsection page}{
  \centering
  \begin{beamercolorbox}[sep=8pt,center]{subsection title}
    \usebeamerfont{subsection title}\insertsubsection\par
  \end{beamercolorbox}
}
% Prevent slide breaks in the middle of a paragraph
\widowpenalties 1 10000
\raggedbottom
\AtBeginPart{
  \frame{\partpage}
}
\AtBeginSection{
  \ifbibliography
  \else
    \frame{\sectionpage}
  \fi
}
\AtBeginSubsection{
  \frame{\subsectionpage}
}
\usepackage{iftex}
\ifPDFTeX
  \usepackage[T1]{fontenc}
  \usepackage[utf8]{inputenc}
  \usepackage{textcomp} % provide euro and other symbols
\else % if luatex or xetex
  \usepackage{unicode-math} % this also loads fontspec
  \defaultfontfeatures{Scale=MatchLowercase}
  \defaultfontfeatures[\rmfamily]{Ligatures=TeX,Scale=1}
\fi
\usepackage{lmodern}

\ifPDFTeX\else
  % xetex/luatex font selection
\fi
% Use upquote if available, for straight quotes in verbatim environments
\IfFileExists{upquote.sty}{\usepackage{upquote}}{}
\IfFileExists{microtype.sty}{% use microtype if available
  \usepackage[]{microtype}
  \UseMicrotypeSet[protrusion]{basicmath} % disable protrusion for tt fonts
}{}
\makeatletter
\@ifundefined{KOMAClassName}{% if non-KOMA class
  \IfFileExists{parskip.sty}{%
    \usepackage{parskip}
  }{% else
    \setlength{\parindent}{0pt}
    \setlength{\parskip}{6pt plus 2pt minus 1pt}}
}{% if KOMA class
  \KOMAoptions{parskip=half}}
\makeatother


\usepackage{longtable,booktabs,array}
\newcounter{none} % for unnumbered tables
\usepackage{calc} % for calculating minipage widths
\usepackage{caption}
% Make caption package work with longtable
\makeatletter
\def\fnum@table{\tablename~\thetable}
\makeatother
\usepackage{graphicx}
\makeatletter
\newsavebox\pandoc@box
\newcommand*\pandocbounded[1]{% scales image to fit in text height/width
  \sbox\pandoc@box{#1}%
  \Gscale@div\@tempa{\textheight}{\dimexpr\ht\pandoc@box+\dp\pandoc@box\relax}%
  \Gscale@div\@tempb{\linewidth}{\wd\pandoc@box}%
  \ifdim\@tempb\p@<\@tempa\p@\let\@tempa\@tempb\fi% select the smaller of both
  \ifdim\@tempa\p@<\p@\scalebox{\@tempa}{\usebox\pandoc@box}%
  \else\usebox{\pandoc@box}%
  \fi%
}
% Set default figure placement to htbp
\def\fps@figure{htbp}
\makeatother


% definitions for citeproc citations
\NewDocumentCommand\citeproctext{}{}
\NewDocumentCommand\citeproc{mm}{%
  \begingroup\def\citeproctext{#2}\cite{#1}\endgroup}
\makeatletter
 % allow citations to break across lines
 \let\@cite@ofmt\@firstofone
 % avoid brackets around text for \cite:
 \def\@biblabel#1{}
 \def\@cite#1#2{{#1\if@tempswa , #2\fi}}
\makeatother
\newlength{\cslhangindent}
\setlength{\cslhangindent}{1.5em}
\newlength{\csllabelwidth}
\setlength{\csllabelwidth}{3em}
\newenvironment{CSLReferences}[2] % #1 hanging-indent, #2 entry-spacing
 {\begin{list}{}{%
  \setlength{\itemindent}{0pt}
  \setlength{\leftmargin}{0pt}
  \setlength{\parsep}{0pt}
  % turn on hanging indent if param 1 is 1
  \ifodd #1
   \setlength{\leftmargin}{\cslhangindent}
   \setlength{\itemindent}{-1\cslhangindent}
  \fi
  % set entry spacing
  \setlength{\itemsep}{#2\baselineskip}}}
 {\end{list}}
\usepackage{calc}
\newcommand{\CSLBlock}[1]{\hfill\break\parbox[t]{\linewidth}{\strut\ignorespaces#1\strut}}
\newcommand{\CSLLeftMargin}[1]{\parbox[t]{\csllabelwidth}{\strut#1\strut}}
\newcommand{\CSLRightInline}[1]{\parbox[t]{\linewidth - \csllabelwidth}{\strut#1\strut}}
\newcommand{\CSLIndent}[1]{\hspace{\cslhangindent}#1}

\ifLuaTeX
\usepackage[bidi=basic,shorthands=off]{babel}
\else
\usepackage[bidi=default,shorthands=off]{babel}
\fi
\ifLuaTeX
  \usepackage{selnolig} % disable illegal ligatures
\fi


\setlength{\emergencystretch}{3em} % prevent overfull lines

\providecommand{\tightlist}{%
  \setlength{\itemsep}{0pt}\setlength{\parskip}{0pt}}



 


\usepackage{standalone}
\usepackage{tikz}
\usepackage{pgfplots}
\usetikzlibrary{decorations.pathreplacing}
\usetikzlibrary{arrows.meta}
\pgfplotsset{compat=1.18}
\makeatletter
\@ifpackageloaded{tcolorbox}{}{\usepackage[skins,breakable]{tcolorbox}}
\@ifpackageloaded{fontawesome5}{}{\usepackage{fontawesome5}}
\definecolor{quarto-callout-color}{HTML}{909090}
\definecolor{quarto-callout-note-color}{HTML}{0758E5}
\definecolor{quarto-callout-important-color}{HTML}{CC1914}
\definecolor{quarto-callout-warning-color}{HTML}{EB9113}
\definecolor{quarto-callout-tip-color}{HTML}{00A047}
\definecolor{quarto-callout-caution-color}{HTML}{FC5300}
\definecolor{quarto-callout-color-frame}{HTML}{acacac}
\definecolor{quarto-callout-note-color-frame}{HTML}{4582ec}
\definecolor{quarto-callout-important-color-frame}{HTML}{d9534f}
\definecolor{quarto-callout-warning-color-frame}{HTML}{f0ad4e}
\definecolor{quarto-callout-tip-color-frame}{HTML}{02b875}
\definecolor{quarto-callout-caution-color-frame}{HTML}{fd7e14}
\makeatother
\makeatletter
\@ifpackageloaded{caption}{}{\usepackage{caption}}
\AtBeginDocument{%
\ifdefined\contentsname
  \renewcommand*\contentsname{Table of contents}
\else
  \newcommand\contentsname{Table of contents}
\fi
\ifdefined\listfigurename
  \renewcommand*\listfigurename{List of Figures}
\else
  \newcommand\listfigurename{List of Figures}
\fi
\ifdefined\listtablename
  \renewcommand*\listtablename{List of Tables}
\else
  \newcommand\listtablename{List of Tables}
\fi
\ifdefined\figurename
  \renewcommand*\figurename{Figure}
\else
  \newcommand\figurename{Figure}
\fi
\ifdefined\tablename
  \renewcommand*\tablename{Table}
\else
  \newcommand\tablename{Table}
\fi
}
\@ifpackageloaded{float}{}{\usepackage{float}}
\floatstyle{ruled}
\@ifundefined{c@chapter}{\newfloat{codelisting}{h}{lop}}{\newfloat{codelisting}{h}{lop}[chapter]}
\floatname{codelisting}{Listing}
\newcommand*\listoflistings{\listof{codelisting}{List of Listings}}
\makeatother
\makeatletter
\makeatother
\makeatletter
\@ifpackageloaded{caption}{}{\usepackage{caption}}
\@ifpackageloaded{subcaption}{}{\usepackage{subcaption}}
\makeatother

\usepackage{bookmark}
\IfFileExists{xurl.sty}{\usepackage{xurl}}{} % add URL line breaks if available
\urlstyle{same}
\hypersetup{
  pdftitle={Lecture 4 - Patents and Intellectual Property Rights},
  pdfauthor={Gerhard Riener},
  pdflang={en-GB},
  hidelinks,
  pdfcreator={LaTeX via pandoc}}


\title{\texorpdfstring{Lecture 4 - Patents and Intellectual Property
Rights}{Lecture 4 - Patents and Intellectual Property Rights}}
\author{Gerhard Riener}
\date{}

\begin{document}
\frame{\titlepage}


\section{Patents and IPRs}\label{patents-and-iprs}

\begin{frame}{IPRs: economic problem}
\protect\phantomsection\label{iprs-economic-problem}
\begin{tcolorbox}[enhanced jigsaw, opacityback=0, colback=white, opacitybacktitle=0.6, coltitle=black, toprule=.15mm, left=2mm, leftrule=.75mm, toptitle=1mm, breakable, bottomtitle=1mm, colbacktitle=quarto-callout-note-color!10!white, titlerule=0mm, title=\textcolor{quarto-callout-note-color}{\faInfo}\hspace{0.5em}{Object of analysis}, arc=.35mm, rightrule=.15mm, bottomrule=.15mm, colframe=quarto-callout-note-color-frame]

IPRs make a non-rival good partially excludable, creating private
incentives for costly innovation.

\end{tcolorbox}

\begin{itemize}
\item
  Knowledge is (partly) \textbf{non-rival} and often \textbf{hard to
  exclude}
\item
  Without protection, imitation can dissipate rents \(\Rightarrow\) weak
  private incentive to incur fixed R\&D cost
\item
  \textbf{Policy objective}

  \begin{itemize}
  \tightlist
  \item
    Provide incentives for invention while limiting static distortions
    in product markets
  \end{itemize}
\end{itemize}
\end{frame}

\begin{frame}{IPRs: instruments}
\protect\phantomsection\label{iprs-instruments}
\begin{itemize}
\tightlist
\item
  Main IPR types:

  \begin{itemize}
  \tightlist
  \item
    Patents
  \item
    Trademarks
  \item
    Copyrights
  \item
    Design rights
  \end{itemize}
\item
  Key design dimensions (patents):

  \begin{itemize}
  \tightlist
  \item
    \textbf{Length} (duration)
  \item
    \textbf{Breadth} (scope)
  \item
    \textbf{Geographical coverage}
  \item
    \textbf{Transferability} (sale, licensing)
  \end{itemize}
\end{itemize}
\end{frame}

\begin{frame}{IPRs: central trade-off}
\protect\phantomsection\label{iprs-central-trade-off}
\begin{itemize}
\item
  Patents create temporary market power \(\Rightarrow\) static
  deadweight loss
\item
  Stronger protection (longer/broader) typically:

  \begin{itemize}
  \tightlist
  \item
    increases expected private returns to R\&D
  \item
    increases the static distortion during protection
  \end{itemize}
\item
  Questions:

  \begin{itemize}
  \tightlist
  \item
    What is an optimal \textbf{length} and \textbf{breadth}?
  \item
    How does competition in R\&D (patent races) affect efficiency?
  \end{itemize}
\end{itemize}
\end{frame}

\section{Scotchmer: ideas model}\label{scotchmer-ideas-model}

\begin{frame}{Ideas model (Scotchmer): primitives}
\protect\phantomsection\label{ideas-model-scotchmer-primitives}
Following Scotchmer (2006)

\begin{itemize}
\item
  An ``idea'' is a pair \((\nu, F)\)
\item
  \(\nu\): per-period consumer surplus under competitive supply (value
  parameter)
\item
  \(F\): fixed cost to develop the idea into an innovation (R\&D cost)
\item
  \textbf{Interpretation}

  \begin{itemize}
  \tightlist
  \item
    \(\nu\) captures the size of social gains from making the idea
    usable
  \item
    \(F\) is the up-front resource cost required for development
  \end{itemize}
\end{itemize}
\end{frame}

\begin{frame}{Ideas model (Scotchmer): social value under discounting}
\protect\phantomsection\label{ideas-model-scotchmer-social-value-under-discounting}
Assume social value lasts forever and the product is competitively
supplied.

\begin{itemize}
\item
  Per-period social value: \(\nu\)
\item
  Discounted social value:
\end{itemize}

\[
\sum_{t=1}^{\infty} \frac{1}{(1+r)^t}\nu = \frac{\nu}{r}
\]

\begin{itemize}
\tightlist
\item
  \textbf{Interpretation}

  \begin{itemize}
  \tightlist
  \item
    Discount rate \(r\) reduces the present value of long-run benefits
  \item
    Longer-lived benefits (lower \(r\)) raise the social value of an
    idea
  \end{itemize}
\end{itemize}
\end{frame}

\begin{frame}{Ideas model (Scotchmer): private returns and deadweight
loss}
\protect\phantomsection\label{ideas-model-scotchmer-private-returns-and-deadweight-loss}
\begin{itemize}
\item
  Firm's per-period private profit under patent: \(\pi \nu\) where
  \(0<\pi<1\)
\item
  Patent profit for discounted length \(T\):
\end{itemize}

\[
\pi \nu T
\]

\begin{itemize}
\item
  Per-period deadweight loss: \(\ell \nu\) \(\Rightarrow\) patent DWL:
  \(\ell \nu T\)
\item
  \textbf{Interpretation}

  \begin{itemize}
  \tightlist
  \item
    \(\pi\) is a reduced-form ``appropriability'' parameter
  \item
    \(\ell\) captures the static distortion created by protection
  \end{itemize}
\end{itemize}
\end{frame}

\begin{frame}{Discounted patent length: \(T\)}
\protect\phantomsection\label{discounted-patent-length-t}
\begin{itemize}
\tightlist
\item
  Let \(\tau\) be the undiscounted duration (in periods)
\item
  Define discounted duration:
\end{itemize}

\[
T = \int_0^{\tau} e^{-rt} \, dt = \frac{1-e^{-r\tau}}{r}
\]

\begin{itemize}
\tightlist
\item
  Discrete-time approximation used in many models:
\end{itemize}

\[
T \approx \sum_{t=1}^{\tau}\frac{1}{(1+r)^t}
\]

\begin{itemize}
\tightlist
\item
  \textbf{Interpretation}

  \begin{itemize}
  \tightlist
  \item
    \(T\) is increasing in \(\tau\) but bounded as \(\tau\to\infty\)
    when \(r>0\)
  \end{itemize}
\end{itemize}
\end{frame}

\begin{frame}{Optimal patent length: innovating firm}
\protect\phantomsection\label{optimal-patent-length-innovating-firm}
\begin{itemize}
\tightlist
\item
  Patent gives discounted net profit:
\end{itemize}

\[
\pi \nu T - F
\]

\begin{itemize}
\item
  Firm invests if \(\pi \nu T \ge F\)
\item
  \textbf{Interpretation}

  \begin{itemize}
  \tightlist
  \item
    Higher \(\nu\) or larger \(\pi\) reduces the minimum protection
    needed for investment
  \item
    Higher \(F\) requires longer (or stronger) protection to break even
  \end{itemize}
\end{itemize}
\end{frame}

\begin{frame}{Optimal patent length: social planner}
\protect\phantomsection\label{optimal-patent-length-social-planner}
\begin{itemize}
\tightlist
\item
  Discounted net social value (invention made):
\end{itemize}

\[
\frac{\nu}{r} - \ell \nu T
\]

\begin{itemize}
\tightlist
\item
  \textbf{Interpretation}

  \begin{itemize}
  \tightlist
  \item
    Planner values the full flow benefit \(\nu\), but counts DWL during
    protection
  \item
    Optimal design trades off inducing investment against static costs
  \end{itemize}
\end{itemize}
\end{frame}

\begin{frame}{Optimal length: heterogeneity and screening (intuition)}
\protect\phantomsection\label{optimal-length-heterogeneity-and-screening-intuition}
\begin{itemize}
\item
  If inventions differ in \((\nu,F)\), ``one-size-fits-all'' length is
  not generally optimal
\item
  Comparative statics (holding other parameters fixed):

  \begin{itemize}
  \tightlist
  \item
    more elastic demand / stronger substitution \(\Rightarrow\) higher
    DWL \(\ell\) \(\Rightarrow\) shorter protection
  \item
    higher development cost \(F\) \(\Rightarrow\) longer protection to
    ensure investment
  \end{itemize}
\end{itemize}
\end{frame}

\begin{frame}{Example: ideas A and B}
\protect\phantomsection\label{example-ideas-a-and-b}
\begin{itemize}
\tightlist
\item
  Idea A: \((\nu_A=5, F_A=10)\)
\item
  Idea B: \((\nu_B=2, F_B=20)\)
\end{itemize}

Let \(T=20\), \(\pi=\tfrac{1}{2}\), \(\ell=\tfrac{1}{4}\),
\(r=\tfrac{1}{3}\).

\begin{itemize}
\tightlist
\item
  Tasks:

  \begin{itemize}
  \tightlist
  \item
    Which ideas are privately profitable (\(\pi \nu T \ge F\))?
  \item
    Which ideas have positive discounted net social value
    \(\left(\frac{\nu}{r}-\ell \nu T - F \ge 0\right)\)?
  \end{itemize}
\item
  \textbf{Interpretation}

  \begin{itemize}
  \tightlist
  \item
    With these parameter values, private profitability need not coincide
    with positive net social value: patent protection can induce
    investment even when net welfare is negative
  \end{itemize}
\end{itemize}
\end{frame}

\section{Breadth: product space}\label{breadth-product-space}

\begin{frame}{Patent breadth: product space (definition)}
\protect\phantomsection\label{patent-breadth-product-space-definition}
\begin{itemize}
\item
  Breadth determines how close a substitute can be without infringing
\item
  Reduced-form implication:

  \begin{itemize}
  \tightlist
  \item
    Narrower breadth \(\Rightarrow\) more close substitutes enter
  \item
    Broader breadth \(\Rightarrow\) fewer close substitutes enter
  \end{itemize}
\item
  \textbf{Interpretation}

  \begin{itemize}
  \tightlist
  \item
    Allowing close substitutes increases the elasticity of demand faced
    by the patent holder
  \end{itemize}
\end{itemize}
\end{frame}

\begin{frame}{Breadth and demand elasticity (intuition)}
\protect\phantomsection\label{breadth-and-demand-elasticity-intuition}
\begin{itemize}
\tightlist
\item
  If close substitutes are allowed:

  \begin{itemize}
  \tightlist
  \item
    residual demand becomes \textbf{more elastic}
  \item
    equilibrium price is lower (all else equal)
  \end{itemize}
\item
  If substitutes are excluded (broader patent):

  \begin{itemize}
  \tightlist
  \item
    residual demand is \textbf{less elastic}
  \item
    equilibrium price is higher (all else equal)
  \end{itemize}
\end{itemize}
\end{frame}

\begin{frame}{Breadth--length trade-off (given a target value)}
\protect\phantomsection\label{breadthlength-trade-off-given-a-target-value}
Assume the ``correct'' expected private value of protection is fixed.

\begin{tcolorbox}[enhanced jigsaw, opacityback=0, colback=white, opacitybacktitle=0.6, coltitle=black, toprule=.15mm, left=2mm, leftrule=.75mm, toptitle=1mm, breakable, bottomtitle=1mm, colbacktitle=quarto-callout-note-color!10!white, titlerule=0mm, title=\textcolor{quarto-callout-note-color}{\faInfo}\hspace{0.5em}{Regimes (product space)}, arc=.35mm, rightrule=.15mm, bottomrule=.15mm, colframe=quarto-callout-note-color-frame]

\begin{itemize}
\tightlist
\item
  Broad--short: \((\hat{T}, \hat{\pi}_1+\hat{\pi}_2)\)
\item
  Narrow--long: \((\tilde{T}, \tilde{\pi}_1)\)
\end{itemize}

\end{tcolorbox}

\begin{itemize}
\tightlist
\item
  Broad patent yields higher per-period profit (includes infringing
  market):
\end{itemize}

\[
\hat{T}(\hat{\pi}_1+\hat{\pi}_2)=\tilde{T}\tilde{\pi}_1
\]

\begin{itemize}
\tightlist
\item
  Therefore:
\end{itemize}

\[
\hat{T}<\tilde{T}
\]

\begin{itemize}
\tightlist
\item
  \textbf{Interpretation}

  \begin{itemize}
  \tightlist
  \item
    Broad protection can be paired with shorter duration to deliver the
    same incentive level
  \end{itemize}
\end{itemize}
\end{frame}

\begin{frame}{Which regime is better?}
\protect\phantomsection\label{which-regime-is-better}
\begin{itemize}
\tightlist
\item
  The best regime depends on substitution patterns:

  \begin{itemize}
  \tightlist
  \item
    substitution between the patented good and an infringing substitute
  \item
    substitution between these goods and the rest of consumption
  \end{itemize}
\item
  \textbf{Interpretation}

  \begin{itemize}
  \tightlist
  \item
    Broad--short: more sensitive to outside substitution (pricing
    alignment across many goods)
  \item
    Narrow--long: more sensitive to within-category substitution
  \end{itemize}
\end{itemize}
\end{frame}

\section{Optimal patent length with endogenous R\&D
(Shy)}\label{optimal-patent-length-with-endogenous-rd-shy}

\begin{frame}{Shy (1995) model: setup}
\protect\phantomsection\label{shyindustrialorganizationtheory1995-model-setup}
\begin{itemize}
\item
  Demand: \(P(Q)=a-Q\)
\item
  Process innovation reduces marginal cost from \(c\) to \(c-x\)
\item
  R\&D effort \(x\) costs \(R(x)\)
\item
  Two-stage game:

  \begin{enumerate}
  \tightlist
  \item
    Regulator chooses patent duration \(\tau\)
  \item
    Firm chooses \(x\) to maximise discounted profit
  \end{enumerate}
\item
  \textbf{Objects}

  \begin{itemize}
  \tightlist
  \item
    Choice variables: \(x\) (firm), \(\tau\) (regulator)
  \item
    Parameters: \(a,c,r\)
  \end{itemize}
\end{itemize}
\end{frame}

\begin{frame}{Shy model: firm's choice of \(x\) given \(\tau\)}
\protect\phantomsection\label{shy-model-firms-choice-of-x-given-tau}
Firm solves:

\[
\max_x \ \Pi(x;\tau)=\sum_{t=1}^{\tau}\rho^{t-1}\pi(x)-R(x),
\quad \rho=\frac{1}{1+r}
\]

Assume:

\begin{itemize}
\item
  per-period profit: \(\pi(x)=(a-c)\,x\)
\item
  cost: \(R(x)=\frac{x^2}{2}\)
\item
  Lemma:
\end{itemize}

\[
\sum_{t=1}^{\tau}\rho^{t-1}=\frac{1-\rho^\tau}{1-\rho}
\]
\end{frame}

\begin{frame}{Shy model: induced innovation level}
\protect\phantomsection\label{shy-model-induced-innovation-level}
FOC implies:

\[
x^I(\tau)=\frac{1-\rho^\tau}{1-\rho}(a-c)
\]

\begin{itemize}
\tightlist
\item
  Comparative statics:

  \begin{itemize}
  \tightlist
  \item
    \(x^I\) increases with \(\tau\)
  \item
    \(x^I\) increases with \(a\) and decreases with \(c\)
  \item
    \(x^I\) increases with \(\rho\) (decreases with \(r\))
  \end{itemize}
\item
  \textbf{Interpretation}

  \begin{itemize}
  \tightlist
  \item
    Longer protection raises the marginal benefit of R\&D because
    profits are earned for more discounted periods
  \end{itemize}
\end{itemize}
\end{frame}

\begin{frame}{Shy model: planner's choice of patent duration
(statement)}
\protect\phantomsection\label{shy-model-planners-choice-of-patent-duration-statement}
Planner chooses \(\tau\) trading off: - higher induced innovation
\(x^I(\tau)\) - static deadweight loss under monopoly pricing during
protection

\begin{itemize}
\item
  Result (as in Shy): optimal duration is finite, \(T^*<\infty\)
\item
  \textbf{Interpretation}

  \begin{itemize}
  \tightlist
  \item
    Marginal benefit of longer protection (higher induced \(x^I(\tau)\))
    eventually falls below the marginal cost (additional monopoly
    distortion during protection)
  \end{itemize}
\end{itemize}
\end{frame}

\section{Patent races}\label{patent-races}

\begin{frame}{Symmetric patent race: setup}
\protect\phantomsection\label{symmetric-patent-race-setup}
\begin{itemize}
\tightlist
\item
  Two symmetric firms may incur a fixed cost \(f\) to establish a
  research division
\item
  Success probability: \(p\) (per firm)
\item
  Payoffs:

  \begin{itemize}
  \tightlist
  \item
    monopoly profit if sole innovator: \(\pi^m\)
  \item
    duopoly profit if both succeed: \(\pi^d\)
  \end{itemize}
\item
  Welfare benchmarks (post-innovation welfare):

  \begin{itemize}
  \tightlist
  \item
    one research division: \(W^m=\pi^m+CS^m\)
  \item
    two research divisions: \(W^d=2\pi^d+CS^d\)
  \end{itemize}
\item
  Assumption (for the comparison):

  \begin{itemize}
  \tightlist
  \item
    \(CS^d>CS^m\) (more competition in the product market raises
    consumer surplus)
  \end{itemize}
\end{itemize}
\end{frame}

\begin{frame}{Patent race: Nash equilibrium condition}
\protect\phantomsection\label{patent-race-nash-equilibrium-condition}
\begin{tcolorbox}[enhanced jigsaw, opacityback=0, colback=white, opacitybacktitle=0.6, coltitle=black, toprule=.15mm, left=2mm, leftrule=.75mm, toptitle=1mm, breakable, bottomtitle=1mm, colbacktitle=quarto-callout-warning-color!10!white, titlerule=0mm, title=\textcolor{quarto-callout-warning-color}{\faExclamationTriangle}\hspace{0.5em}{Duplication incentive}, arc=.35mm, rightrule=.15mm, bottomrule=.15mm, colframe=quarto-callout-warning-color-frame]

``Winner-takes-all'' payoffs can create privately excessive entry into
R\&D when firms ignore duplication costs.

\end{tcolorbox}

\begin{itemize}
\tightlist
\item
  Firms choose \(I\) or \(NI\) simultaneously
\item
  \((I,I)\) is a Nash equilibrium if:
\end{itemize}

\[
f \le p(1-p)\pi^m + p^2\pi^d \equiv f^{priv}_2
\]

\begin{itemize}
\tightlist
\item
  \textbf{Interpretation}

  \begin{itemize}
  \tightlist
  \item
    Private incentives include the probability of being the unique
    winner and the case where both succeed
  \end{itemize}
\end{itemize}
\end{frame}

\begin{frame}{Patent race: social optimum condition}
\protect\phantomsection\label{patent-race-social-optimum-condition}
\begin{itemize}
\tightlist
\item
  It is socially optimal to have one research division rather than two
  if:
\end{itemize}

\[
f \ge p(1-2p)W^m + p^2W^d \equiv f^{publ}_2
\]

\begin{itemize}
\tightlist
\item
  \textbf{Interpretation}

  \begin{itemize}
  \tightlist
  \item
    The planner compares expected welfare under one vs two research
    divisions, counting duplication cost \(f\)
  \end{itemize}
\end{itemize}
\end{frame}

\begin{frame}{Socially excessive R\&D (region)}
\protect\phantomsection\label{socially-excessive-rd-region}
\begin{itemize}
\tightlist
\item
  Socially excessive duplication occurs when:
\end{itemize}

\[
f^{publ}_2 < f < f^{priv}_2
\]

\begin{itemize}
\tightlist
\item
  Interpretation:

  \begin{itemize}
  \tightlist
  \item
    Firms overinvest when the negative externality on rivals' profits
    outweighs the consumer-surplus gain from having two innovators
  \end{itemize}
\end{itemize}
\end{frame}

\begin{frame}{Summary and next week}
\protect\phantomsection\label{summary-and-next-week}
\textbf{Summary}

\begin{itemize}
\tightlist
\item
  Patents trade off dynamic incentives against static distortions
  (deadweight loss during protection)
\item
  In the ideas model, investment requires \(\pi \nu T \ge F\), while
  welfare accounts for \(\frac{\nu}{r}\) and the DWL term \(\ell \nu T\)
\item
  Breadth and length can be substitutes in delivering a given private
  incentive level (broad--short vs narrow--long)
\item
  Patent races can generate socially excessive duplication when private
  entry incentives exceed social benefits
\end{itemize}

\textbf{Next week: multi-stage games}

\begin{itemize}
\tightlist
\item
  Commitment and first-mover advantage (Stackelberg)
\item
  Subgame perfect equilibrium and backward induction
\item
  Strategic delegation (Vickers)
\end{itemize}
\end{frame}

\begin{frame}{References}
\protect\phantomsection\label{references}
\protect\phantomsection\label{refs}
\begin{CSLReferences}{1}{1}
\bibitem[\citeproctext]{ref-scotchmerInnovationIncentives2006}
Scotchmer, Suzanne. 2006. \emph{Innovation and Incentives}. MIT Press.

\bibitem[\citeproctext]{ref-shyIndustrialOrganizationTheory1995}
Shy, Oz. 1995. \emph{Industrial Organization: Theory and Applications}.
MIT Press.

\end{CSLReferences}
\end{frame}




\end{document}
