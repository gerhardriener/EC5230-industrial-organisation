% Options for packages loaded elsewhere
% Options for packages loaded elsewhere
\PassOptionsToPackage{unicode}{hyperref}
\PassOptionsToPackage{hyphens}{url}
%
\documentclass[
  ignorenonframetext,
  british,
]{beamer}
\newif\ifbibliography
\usepackage{pgfpages}
\setbeamertemplate{caption}[numbered]
\setbeamertemplate{caption label separator}{: }
\setbeamercolor{caption name}{fg=normal text.fg}
\beamertemplatenavigationsymbolsempty
% remove section numbering
\setbeamertemplate{part page}{
  \centering
  \begin{beamercolorbox}[sep=16pt,center]{part title}
    \usebeamerfont{part title}\insertpart\par
  \end{beamercolorbox}
}
\setbeamertemplate{section page}{
  \centering
  \begin{beamercolorbox}[sep=12pt,center]{section title}
    \usebeamerfont{section title}\insertsection\par
  \end{beamercolorbox}
}
\setbeamertemplate{subsection page}{
  \centering
  \begin{beamercolorbox}[sep=8pt,center]{subsection title}
    \usebeamerfont{subsection title}\insertsubsection\par
  \end{beamercolorbox}
}
% Prevent slide breaks in the middle of a paragraph
\widowpenalties 1 10000
\raggedbottom
\AtBeginPart{
  \frame{\partpage}
}
\AtBeginSection{
  \ifbibliography
  \else
    \frame{\sectionpage}
  \fi
}
\AtBeginSubsection{
  \frame{\subsectionpage}
}
\usepackage{iftex}
\ifPDFTeX
  \usepackage[T1]{fontenc}
  \usepackage[utf8]{inputenc}
  \usepackage{textcomp} % provide euro and other symbols
\else % if luatex or xetex
  \usepackage{unicode-math} % this also loads fontspec
  \defaultfontfeatures{Scale=MatchLowercase}
  \defaultfontfeatures[\rmfamily]{Ligatures=TeX,Scale=1}
\fi
\usepackage{lmodern}

\ifPDFTeX\else
  % xetex/luatex font selection
\fi
% Use upquote if available, for straight quotes in verbatim environments
\IfFileExists{upquote.sty}{\usepackage{upquote}}{}
\IfFileExists{microtype.sty}{% use microtype if available
  \usepackage[]{microtype}
  \UseMicrotypeSet[protrusion]{basicmath} % disable protrusion for tt fonts
}{}
\makeatletter
\@ifundefined{KOMAClassName}{% if non-KOMA class
  \IfFileExists{parskip.sty}{%
    \usepackage{parskip}
  }{% else
    \setlength{\parindent}{0pt}
    \setlength{\parskip}{6pt plus 2pt minus 1pt}}
}{% if KOMA class
  \KOMAoptions{parskip=half}}
\makeatother


\usepackage{longtable,booktabs,array}
\newcounter{none} % for unnumbered tables
\usepackage{calc} % for calculating minipage widths
\usepackage{caption}
% Make caption package work with longtable
\makeatletter
\def\fnum@table{\tablename~\thetable}
\makeatother
\usepackage{graphicx}
\makeatletter
\newsavebox\pandoc@box
\newcommand*\pandocbounded[1]{% scales image to fit in text height/width
  \sbox\pandoc@box{#1}%
  \Gscale@div\@tempa{\textheight}{\dimexpr\ht\pandoc@box+\dp\pandoc@box\relax}%
  \Gscale@div\@tempb{\linewidth}{\wd\pandoc@box}%
  \ifdim\@tempb\p@<\@tempa\p@\let\@tempa\@tempb\fi% select the smaller of both
  \ifdim\@tempa\p@<\p@\scalebox{\@tempa}{\usebox\pandoc@box}%
  \else\usebox{\pandoc@box}%
  \fi%
}
% Set default figure placement to htbp
\def\fps@figure{htbp}
\makeatother


% definitions for citeproc citations
\NewDocumentCommand\citeproctext{}{}
\NewDocumentCommand\citeproc{mm}{%
  \begingroup\def\citeproctext{#2}\cite{#1}\endgroup}
\makeatletter
 % allow citations to break across lines
 \let\@cite@ofmt\@firstofone
 % avoid brackets around text for \cite:
 \def\@biblabel#1{}
 \def\@cite#1#2{{#1\if@tempswa , #2\fi}}
\makeatother
\newlength{\cslhangindent}
\setlength{\cslhangindent}{1.5em}
\newlength{\csllabelwidth}
\setlength{\csllabelwidth}{3em}
\newenvironment{CSLReferences}[2] % #1 hanging-indent, #2 entry-spacing
 {\begin{list}{}{%
  \setlength{\itemindent}{0pt}
  \setlength{\leftmargin}{0pt}
  \setlength{\parsep}{0pt}
  % turn on hanging indent if param 1 is 1
  \ifodd #1
   \setlength{\leftmargin}{\cslhangindent}
   \setlength{\itemindent}{-1\cslhangindent}
  \fi
  % set entry spacing
  \setlength{\itemsep}{#2\baselineskip}}}
 {\end{list}}
\usepackage{calc}
\newcommand{\CSLBlock}[1]{\hfill\break\parbox[t]{\linewidth}{\strut\ignorespaces#1\strut}}
\newcommand{\CSLLeftMargin}[1]{\parbox[t]{\csllabelwidth}{\strut#1\strut}}
\newcommand{\CSLRightInline}[1]{\parbox[t]{\linewidth - \csllabelwidth}{\strut#1\strut}}
\newcommand{\CSLIndent}[1]{\hspace{\cslhangindent}#1}

\ifLuaTeX
\usepackage[bidi=basic,shorthands=off]{babel}
\else
\usepackage[bidi=default,shorthands=off]{babel}
\fi
\ifLuaTeX
  \usepackage{selnolig} % disable illegal ligatures
\fi


\setlength{\emergencystretch}{3em} % prevent overfull lines

\providecommand{\tightlist}{%
  \setlength{\itemsep}{0pt}\setlength{\parskip}{0pt}}



 


\usepackage{standalone}
\usepackage{tikz}
\usepackage{pgfplots}
\usetikzlibrary{decorations.pathreplacing}
\usetikzlibrary{arrows.meta}
\pgfplotsset{compat=1.18}
\makeatletter
\@ifpackageloaded{tcolorbox}{}{\usepackage[skins,breakable]{tcolorbox}}
\@ifpackageloaded{fontawesome5}{}{\usepackage{fontawesome5}}
\definecolor{quarto-callout-color}{HTML}{909090}
\definecolor{quarto-callout-note-color}{HTML}{0758E5}
\definecolor{quarto-callout-important-color}{HTML}{CC1914}
\definecolor{quarto-callout-warning-color}{HTML}{EB9113}
\definecolor{quarto-callout-tip-color}{HTML}{00A047}
\definecolor{quarto-callout-caution-color}{HTML}{FC5300}
\definecolor{quarto-callout-color-frame}{HTML}{acacac}
\definecolor{quarto-callout-note-color-frame}{HTML}{4582ec}
\definecolor{quarto-callout-important-color-frame}{HTML}{d9534f}
\definecolor{quarto-callout-warning-color-frame}{HTML}{f0ad4e}
\definecolor{quarto-callout-tip-color-frame}{HTML}{02b875}
\definecolor{quarto-callout-caution-color-frame}{HTML}{fd7e14}
\makeatother
\makeatletter
\@ifpackageloaded{caption}{}{\usepackage{caption}}
\AtBeginDocument{%
\ifdefined\contentsname
  \renewcommand*\contentsname{Table of contents}
\else
  \newcommand\contentsname{Table of contents}
\fi
\ifdefined\listfigurename
  \renewcommand*\listfigurename{List of Figures}
\else
  \newcommand\listfigurename{List of Figures}
\fi
\ifdefined\listtablename
  \renewcommand*\listtablename{List of Tables}
\else
  \newcommand\listtablename{List of Tables}
\fi
\ifdefined\figurename
  \renewcommand*\figurename{Figure}
\else
  \newcommand\figurename{Figure}
\fi
\ifdefined\tablename
  \renewcommand*\tablename{Table}
\else
  \newcommand\tablename{Table}
\fi
}
\@ifpackageloaded{float}{}{\usepackage{float}}
\floatstyle{ruled}
\@ifundefined{c@chapter}{\newfloat{codelisting}{h}{lop}}{\newfloat{codelisting}{h}{lop}[chapter]}
\floatname{codelisting}{Listing}
\newcommand*\listoflistings{\listof{codelisting}{List of Listings}}
\makeatother
\makeatletter
\makeatother
\makeatletter
\@ifpackageloaded{caption}{}{\usepackage{caption}}
\@ifpackageloaded{subcaption}{}{\usepackage{subcaption}}
\makeatother

\usepackage{bookmark}
\IfFileExists{xurl.sty}{\usepackage{xurl}}{} % add URL line breaks if available
\urlstyle{same}
\hypersetup{
  pdftitle={Lecture 3 - Innovation},
  pdfauthor={Gerhard Riener},
  pdflang={en-GB},
  hidelinks,
  pdfcreator={LaTeX via pandoc}}


\title{\texorpdfstring{Lecture 3 - Innovation}{Lecture 3 - Innovation}}
\author{Gerhard Riener}
\date{}

\begin{document}
\frame{\titlepage}


\section{Innovation and Market
Structure}\label{innovation-and-market-structure}

\begin{frame}{Innovation as a margin of competition}
\protect\phantomsection\label{innovation-as-a-margin-of-competition}
Innovation is the \textbf{dynamic} margin of competition: firms try to
change the game by shifting \textbf{costs} and \textbf{demand}.

\textbf{Why IO cares}

\begin{itemize}
\tightlist
\item
  Innovation changes \textbf{prices, mark-ups, and welfare} today
\item
  Innovation changes \textbf{market structure} tomorrow (entry/exit,
  dominance, concentration)
\item
  Policy trade-off: \textbf{static efficiency} (low prices) vs
  \textbf{dynamic efficiency} (strong incentives)
\end{itemize}
\end{frame}

\begin{frame}{Types of innovation (and what we model today)}
\protect\phantomsection\label{types-of-innovation-and-what-we-model-today}
\textbf{Process innovation} (this lecture's workhorse case)

\begin{itemize}
\tightlist
\item
  Lowers marginal (or average) cost, e.g.~a better production method
\item
  Think: a shift from \(c_0\) to \(c_1<c_0\)
\end{itemize}

\textbf{Product innovation}

\begin{itemize}
\tightlist
\item
  Raises willingness to pay, expands demand, or creates new varieties
\item
  Often analysed with differentiated products and quality ladders (later
  in the course)
\end{itemize}

\textbf{Stages (high level):} Research → Development → Adoption
\end{frame}

\begin{frame}{Learning objectives}
\protect\phantomsection\label{learning-objectives}
\begin{itemize}
\tightlist
\item
  Compute the \textbf{private value} of a process innovation under
  different market structures
\item
  Define \textbf{drastic vs non-drastic} innovations and explain the
  \textbf{replacement effect} (Arrow 1972)
\item
  Understand how \textbf{free entry + endogenous R\&D} link innovation
  to \textbf{concentration} (Dasgupta and Stiglitz 1980)
\end{itemize}

\begin{tcolorbox}[enhanced jigsaw, opacitybacktitle=0.6, colbacktitle=quarto-callout-note-color!10!white, breakable, bottomrule=.15mm, rightrule=.15mm, colframe=quarto-callout-note-color-frame, colback=white, bottomtitle=1mm, toprule=.15mm, opacityback=0, title=\textcolor{quarto-callout-note-color}{\faInfo}\hspace{0.5em}{Today: roadmap}, titlerule=0mm, arc=.35mm, coltitle=black, toptitle=1mm, leftrule=.75mm, left=2mm]

\begin{enumerate}
\tightlist
\item
  Benchmark model: value under \textbf{monopoly}, \textbf{perfect
  competition}, and \textbf{social planner}
\item
  \textbf{Drastic vs non-drastic} innovations and the
  \textbf{replacement effect}
\item
  Innovation with \textbf{oligopoly} and \textbf{entry threat}
\item
  \textbf{Concentration} and \textbf{R\&D} (Dasgupta and Stiglitz 1980)
\end{enumerate}

\end{tcolorbox}
\end{frame}

\begin{frame}{Discussion: Why does innovation matter for growth?}
\protect\phantomsection\label{discussion-why-does-innovation-matter-for-growth}
Before the Industrial Revolution (\textasciitilde1760--1840), global GDP
per capita was roughly flat for centuries. Since then, it has grown
exponentially.

\begin{tcolorbox}[enhanced jigsaw, opacitybacktitle=0.6, colbacktitle=quarto-callout-note-color!10!white, breakable, bottomrule=.15mm, rightrule=.15mm, colframe=quarto-callout-note-color-frame, colback=white, bottomtitle=1mm, toprule=.15mm, opacityback=0, title=\textcolor{quarto-callout-note-color}{\faInfo}\hspace{0.5em}{Question for you}, titlerule=0mm, arc=.35mm, coltitle=black, toptitle=1mm, leftrule=.75mm, left=2mm]

\begin{itemize}
\tightlist
\item
  What role did innovation (new production methods, machinery,
  transport) play in this transformation?
\item
  Why might market structure affect the \emph{rate} of innovation ---
  and therefore long-run growth?
\item
  Should we expect monopolies or competitive markets to innovate more?
  (We'll answer this formally in a moment.)
\end{itemize}

\end{tcolorbox}
\end{frame}

\begin{frame}{Market structure and innovation: a two-way relationship}
\protect\phantomsection\label{market-structure-and-innovation-a-two-way-relationship}
\textbf{Market structure → innovation}

\begin{itemize}
\tightlist
\item
  Competition affects profits, appropriability, and the gain from
  becoming ``better'' than rivals
\end{itemize}

\textbf{Innovation → market structure}

\begin{itemize}
\tightlist
\item
  Cost/demand shifts affect entry, market shares, and concentration (and
  can create dominance)
\end{itemize}
\end{frame}

\begin{frame}{Measuring incentives: willingness to pay (WTP)}
\protect\phantomsection\label{measuring-incentives-willingness-to-pay-wtp}
\begin{itemize}
\tightlist
\item
  \textbf{Firm WTP}: the max lump-sum payment that leaves profits
  unchanged \(\Rightarrow \ \text{WTP} = \Delta \pi\)
\item
  \textbf{Planner WTP}: the max lump-sum payment that leaves welfare
  unchanged \(\Rightarrow \ \text{WTP} = \Delta W\)
\end{itemize}

In what follows we compute WTP for: \textbf{Monopoly} (before/after),
\textbf{perfect competition} (before; exclusive rights after), and the
\textbf{social planner}.
\end{frame}

\section{Benchmarks: Value of a Process
Innovation}\label{benchmarks-value-of-a-process-innovation}

\begin{frame}{Setup: linear demand + cost-reducing innovation}
\protect\phantomsection\label{setup-linear-demand-cost-reducing-innovation}
\textbf{Inverse demand}

\[
P(Q) = A - Q
\]

\textbf{Technology}

\begin{itemize}
\tightlist
\item
  Constant marginal cost \(c \in \{c_0, c_1\}\)
\item
  Process innovation reduces marginal cost from \(c_0\) to \(c_1\) with
  \(c_1 < c_0 < A\)
\end{itemize}

\begin{tcolorbox}[enhanced jigsaw, opacitybacktitle=0.6, colbacktitle=quarto-callout-note-color!10!white, breakable, bottomrule=.15mm, rightrule=.15mm, colframe=quarto-callout-note-color-frame, colback=white, bottomtitle=1mm, toprule=.15mm, opacityback=0, title=\textcolor{quarto-callout-note-color}{\faInfo}\hspace{0.5em}{Goal}, titlerule=0mm, arc=.35mm, coltitle=black, toptitle=1mm, leftrule=.75mm, left=2mm]

Compute the value of moving from \(c_0\) to \(c_1\) under different
market structures.

\end{tcolorbox}
\end{frame}

\begin{frame}{Monopoly benchmark: problem and solution}
\protect\phantomsection\label{monopoly-benchmark-problem-and-solution}
\textbf{Problem}

\[
\max_Q \ (A - Q - c)Q
\]

\textbf{Solution}

\[
Q^m(c) = \frac{A-c}{2}, \qquad P^m(c) = \frac{A+c}{2}, \qquad \pi^m(c) = \frac{(A-c)^2}{4}
\]
\end{frame}

\begin{frame}{Monopoly WTP for innovation}
\protect\phantomsection\label{monopoly-wtp-for-innovation}
\textbf{WTP for the innovation}

\[
\Delta \pi^m = \pi^m(c_1) - \pi^m(c_0) = \frac{(A-c_1)^2 - (A-c_0)^2}{4}
\]

\begin{tcolorbox}[enhanced jigsaw, opacitybacktitle=0.6, colbacktitle=quarto-callout-note-color!10!white, breakable, bottomrule=.15mm, rightrule=.15mm, colframe=quarto-callout-note-color-frame, colback=white, bottomtitle=1mm, toprule=.15mm, opacityback=0, title=\textcolor{quarto-callout-note-color}{\faInfo}\hspace{0.5em}{Interpretation}, titlerule=0mm, arc=.35mm, coltitle=black, toptitle=1mm, leftrule=.75mm, left=2mm]

\(\Delta \pi^m\) is the monopolist's value of \emph{exclusive} access to
the lower cost \(c_1\) (it is incremental because the firm already earns
rents at \(c_0\)).

\end{tcolorbox}
\end{frame}

\begin{frame}{Perfect competition benchmark (innovation creates rents)}
\protect\phantomsection\label{perfect-competition-benchmark-innovation-creates-rents}
\textbf{Before innovation (all firms at MC \(c_0\)):}

\begin{itemize}
\tightlist
\item
  Competitive price: \(P^{pc}_0 = c_0\)
\item
  Firm profit: \(\pi = 0\)
\end{itemize}

\textbf{After innovation (innovator has exclusive use / patent):}

\begin{itemize}
\tightlist
\item
  \textbf{Drastic}: innovator behaves as a monopolist with cost \(c_1\)
\item
  \textbf{Non-drastic}: innovator is constrained by the competitive
  fringe at cost \(c_0\) (limit pricing)
\end{itemize}
\end{frame}

\begin{frame}{Innovation WTP under perfect competition}
\protect\phantomsection\label{innovation-wtp-under-perfect-competition}
\textbf{Innovator profit / WTP}

\[
\Delta \pi^{pc} =
\begin{cases}
\pi^m(c_1) & \text{(drastic)}\\
(c_0 - c_1)(A - c_0) & \text{(non-drastic, } p=c_0 \text{)}
\end{cases}
\]

\textbf{Why the non-drastic formula?} If the innovator sets \(p=c_0\),
quantity is \(Q=A-c_0\), so profits are \((p-c_1)Q=(c_0-c_1)(A-c_0)\).
\end{frame}

\begin{frame}{Drastic vs non-drastic innovation condition}
\protect\phantomsection\label{drastic-vs-non-drastic-innovation-condition}
\[
\text{Drastic: } \;
P^m(c_1) < c_0
\;\Leftrightarrow\;
\frac{A+c_1}{2} < c_0
\;\Leftrightarrow\;
A + c_1 < 2c_0
\]

\[
\text{Non-drastic: } \;
P^m(c_1) \ge c_0
\;\Leftrightarrow\;
\frac{A+c_1}{2} \ge c_0
\;\Leftrightarrow\;
A + c_1 \ge 2c_0
\]

Boundary case: \(P^m(c_1)=c_0\) (equivalently \(A+c_1=2c_0\)).

\resizebox{0.95\textwidth}{!}{\includestandalone{../figs/source/fig-innovation-drastic-threshold}}
\end{frame}

\begin{frame}{Numeric check: drastic or non-drastic?}
\protect\phantomsection\label{numeric-check-drastic-or-non-drastic}
Use the same values as panel (a): \(A=18\), \(c_0=10\), and \(c_1=6\).

\begin{itemize}
\tightlist
\item
  \textbf{Drastic condition}: \(A + c_1 = 24 > 2c_0 = 20\) →
  \textbf{non-drastic}
\item
  Innovator limit-prices at \(p = c_0 = 10\), sells \(Q = A - c_0 = 8\)
\item
  Competitive innovator profit:
  \((c_0 - c_1)(A - c_0) = 4 \times 8 = 32\)
\item
  Recall monopoly WTP:
  \(\Delta\pi^m = \frac{(12)^2 - (8)^2}{4} = \frac{144 - 64}{4} = 20\)
\end{itemize}

\begin{tcolorbox}[enhanced jigsaw, opacitybacktitle=0.6, colbacktitle=quarto-callout-note-color!10!white, breakable, bottomrule=.15mm, rightrule=.15mm, colframe=quarto-callout-note-color-frame, colback=white, bottomtitle=1mm, toprule=.15mm, opacityback=0, title=\textcolor{quarto-callout-note-color}{\faInfo}\hspace{0.5em}{Replacement effect preview}, titlerule=0mm, arc=.35mm, coltitle=black, toptitle=1mm, leftrule=.75mm, left=2mm]

The competitive innovator earns \(32 > 20\) (the monopolist's WTP). The
monopolist gains less because it already earns \(\pi^m(c_0) = 16\)
before innovating --- it is partly \textbf{replacing} existing rents.

\end{tcolorbox}
\end{frame}

\begin{frame}{Replacement effect: visual comparison}
\protect\phantomsection\label{replacement-effect-visual-comparison}
\begin{itemize}
\tightlist
\item
  Left panel: the incumbent monopolist already earns \(\pi^m(c_0)\)
  before innovating.
\item
  Innovation raises monopoly profit only by the increment
  \(\pi^m(c_1)-\pi^m(c_0)\).
\item
  Right panel: under competitive pre-innovation conditions, the
  innovation mostly \textbf{creates} rents.
\end{itemize}

\resizebox{0.95\textwidth}{!}{\includestandalone{../figs/source/fig-innovation-replacement-effect}}
\end{frame}

\begin{frame}{Replacement effect (Arrow 1972)}
\protect\phantomsection\label{replacement-effect-arroweconomicwelfareallocation1972}
\begin{itemize}
\tightlist
\item
  The key object is the \textbf{incremental} value of innovation.
\item
  Competitive benchmark: pre-innovation profits are approximately zero,
  so \(\Delta\pi^{pc} \approx \pi_{\text{after}}^{pc}\).
\item
  Monopoly benchmark: pre-innovation profits are positive, so
  \(\Delta\pi^{m} = \pi_{\text{after}}^{m} - \pi^{m}(c_0)\).
\item
  This gap in incremental value is the \textbf{replacement effect}.
\end{itemize}
\end{frame}

\begin{frame}{Drastic benchmark: one-line wedge}
\protect\phantomsection\label{drastic-benchmark-one-line-wedge}
If innovation is drastic and grants exclusive rights:

\[
\Delta \pi^{pc} = \pi^m(c_1), \qquad \Delta \pi^m = \pi^m(c_1) - \pi^m(c_0)
\]

\[
\Delta \pi^{pc} - \Delta \pi^m = \pi^m(c_0) > 0
\]

\begin{itemize}
\tightlist
\item
  The wedge equals pre-innovation monopoly rents, \(\pi^m(c_0)\).
\item
  The replacement effect is strongest when baseline monopoly rents are
  high.
\end{itemize}
\end{frame}

\begin{frame}{Replacement effect: intuition}
\protect\phantomsection\label{replacement-effect-intuition}
\begin{tcolorbox}[enhanced jigsaw, opacitybacktitle=0.6, colbacktitle=quarto-callout-important-color!10!white, breakable, bottomrule=.15mm, rightrule=.15mm, colframe=quarto-callout-important-color-frame, colback=white, bottomtitle=1mm, toprule=.15mm, opacityback=0, title=\textcolor{quarto-callout-important-color}{\faExclamation}\hspace{0.5em}{Takeaway}, titlerule=0mm, arc=.35mm, coltitle=black, toptitle=1mm, leftrule=.75mm, left=2mm]

Holding everything else fixed, an incumbent monopolist has
\textbf{weaker} incentives to innovate than a competitive industry
because it is ``replacing'' its own pre-innovation rents.

\end{tcolorbox}
\end{frame}

\begin{frame}{Appropriability and imitation}
\protect\phantomsection\label{appropriability-and-imitation}
\begin{itemize}
\tightlist
\item
  If the innovation cannot be protected (instant diffusion / perfect
  imitation), competition drives price to \(c_1\) and the innovator's
  WTP is zero.
\item
  Most IO policy questions start from imperfect but positive
  appropriability (patents, secrecy, lead time).
\end{itemize}
\end{frame}

\begin{frame}{Empirical example: competition and innovation}
\protect\phantomsection\label{empirical-example-competition-and-innovation}
\begin{itemize}
\tightlist
\item
  Low competition: firms have little pressure to ``escape.''
\item
  Intermediate competition: innovation incentives are strongest for
  neck-and-neck firms.
\item
  Very high competition: post-innovation rents are compressed, reducing
  innovation incentives.
\end{itemize}

\begin{tcolorbox}[enhanced jigsaw, opacitybacktitle=0.6, colbacktitle=quarto-callout-note-color!10!white, breakable, bottomrule=.15mm, rightrule=.15mm, colframe=quarto-callout-note-color-frame, colback=white, bottomtitle=1mm, toprule=.15mm, opacityback=0, title=\textcolor{quarto-callout-note-color}{\faInfo}\hspace{0.5em}{Inverted-U evidence (QJE)}, titlerule=0mm, arc=.35mm, coltitle=black, toptitle=1mm, leftrule=.75mm, left=2mm]

Aghion et al. (2005) document an \textbf{inverted-U} relationship
between product-market competition and innovation (patents/citations) in
UK firms, consistent with ``escape competition'' incentives for
neck-and-neck firms and a \textbf{replacement effect} for laggards.

\end{tcolorbox}

\resizebox{0.8\textwidth}{!}{\includestandalone{../figs/source/fig-innovation-inverted-u}}
\end{frame}

\section{Social Value of Innovation}\label{social-value-of-innovation}

\begin{frame}{Social planner: value of a cost reduction}
\protect\phantomsection\label{social-planner-value-of-a-cost-reduction}
\textbf{Efficient output} (price equals marginal cost)

\[
Q^{sp}(c) = A - c
\]

\textbf{Total surplus (welfare):}

\[
W(c) = \int_0^{A-c} (A - Q - c) \, dQ = \frac{(A-c)^2}{2}
\]

\begin{tcolorbox}[enhanced jigsaw, opacitybacktitle=0.6, colbacktitle=quarto-callout-note-color!10!white, breakable, bottomrule=.15mm, rightrule=.15mm, colframe=quarto-callout-note-color-frame, colback=white, bottomtitle=1mm, toprule=.15mm, opacityback=0, title=\textcolor{quarto-callout-note-color}{\faInfo}\hspace{0.5em}{Interpretation}, titlerule=0mm, arc=.35mm, coltitle=black, toptitle=1mm, leftrule=.75mm, left=2mm]

In this linear example, \(\Delta W = 2\,\Delta \pi^m\): the planner
values the output expansion that the monopolist does not internalise.

\end{tcolorbox}
\end{frame}

\begin{frame}{Social planner WTP for innovation}
\protect\phantomsection\label{social-planner-wtp-for-innovation}
\textbf{Max WTP for innovation (social planner):}

\[
\Delta W = W(c_1) - W(c_0) = \frac{(A-c_1)^2 - (A-c_0)^2}{2}
\]

\begin{itemize}
\tightlist
\item
  This is the full increase in total surplus from lowering marginal
  cost.
\item
  In the linear benchmark, it is exactly twice the monopoly incremental
  profit.
\end{itemize}
\end{frame}

\begin{frame}{Putting the benchmarks side-by-side}
\protect\phantomsection\label{putting-the-benchmarks-side-by-side}
{\def\LTcaptype{none} % do not increment counter
\begin{longtable}[]{@{}
  >{\raggedright\arraybackslash}p{(\linewidth - 4\tabcolsep) * \real{0.3333}}
  >{\raggedright\arraybackslash}p{(\linewidth - 4\tabcolsep) * \real{0.3333}}
  >{\raggedright\arraybackslash}p{(\linewidth - 4\tabcolsep) * \real{0.3333}}@{}}
\toprule\noalign{}
\begin{minipage}[b]{\linewidth}\raggedright
Environment
\end{minipage} & \begin{minipage}[b]{\linewidth}\raggedright
What the innovator captures
\end{minipage} & \begin{minipage}[b]{\linewidth}\raggedright
WTP / value
\end{minipage} \\
\midrule\noalign{}
\endhead
Monopoly (incumbent) & Incremental rents &
\(\Delta\pi^m=\pi^m(c_1)-\pi^m(c_0)\) \\
Perfect competition (pre) + exclusive rights (post) & Mostly created
rents & \(\Delta\pi^{pc}=\pi^m(c_1)\) if drastic; otherwise limit
pricing profit \\
Social planner & Full surplus gain & \(\Delta W=W(c_1)-W(c_0)\) \\
\bottomrule\noalign{}
\end{longtable}
}

\begin{tcolorbox}[enhanced jigsaw, opacitybacktitle=0.6, colbacktitle=quarto-callout-important-color!10!white, breakable, bottomrule=.15mm, rightrule=.15mm, colframe=quarto-callout-important-color-frame, colback=white, bottomtitle=1mm, toprule=.15mm, opacityback=0, title=\textcolor{quarto-callout-important-color}{\faExclamation}\hspace{0.5em}{Policy interpretation}, titlerule=0mm, arc=.35mm, coltitle=black, toptitle=1mm, leftrule=.75mm, left=2mm]

Private incentives and social value need not align: \textbf{\(\Delta W\)
can exceed \(\Delta\pi\)}, but market power created by IPRs also
generates static distortions. This is the basic patents trade-off.

\end{tcolorbox}
\end{frame}

\begin{frame}{Empirical example: private vs social returns}
\protect\phantomsection\label{empirical-example-private-vs-social-returns}
\begin{itemize}
\tightlist
\item
  Measured producer gains capture only part of innovation's value.
\item
  Consumer surplus from quality improvements can be very large.
\item
  This gap motivates policy tools that raise private appropriability.
\end{itemize}

\begin{tcolorbox}[enhanced jigsaw, opacitybacktitle=0.6, colbacktitle=quarto-callout-note-color!10!white, breakable, bottomrule=.15mm, rightrule=.15mm, colframe=quarto-callout-note-color-frame, colback=white, bottomtitle=1mm, toprule=.15mm, opacityback=0, title=\textcolor{quarto-callout-note-color}{\faInfo}\hspace{0.5em}{Welfare gains from product innovation (JPE)}, titlerule=0mm, arc=.35mm, coltitle=black, toptitle=1mm, leftrule=.75mm, left=2mm]

Trajtenberg (1989) estimates consumer and producer surplus gains from
quality improvements in computed tomography (CT) scanners, illustrating
that the \textbf{social value} of innovation can greatly exceed the
innovator's private returns.

\end{tcolorbox}
\end{frame}

\begin{frame}{Quick numeric check (optional)}
\protect\phantomsection\label{quick-numeric-check-optional}
Take \(A=20\), \(c_0=10\), and \(c_1=6\).

\begin{itemize}
\tightlist
\item
  Monopoly: \(\pi^m(c_0)=25\), \(\pi^m(c_1)=49\), so \(\Delta\pi^m=24\)
\item
  Planner: \(W(c_0)=50\), \(W(c_1)=98\), so \(\Delta W=48\) (twice the
  monopoly gain)
\end{itemize}

\begin{tcolorbox}[enhanced jigsaw, opacitybacktitle=0.6, colbacktitle=quarto-callout-note-color!10!white, breakable, bottomrule=.15mm, rightrule=.15mm, colframe=quarto-callout-note-color-frame, colback=white, bottomtitle=1mm, toprule=.15mm, opacityback=0, title=\textcolor{quarto-callout-note-color}{\faInfo}\hspace{0.5em}{Discussion question}, titlerule=0mm, arc=.35mm, coltitle=black, toptitle=1mm, leftrule=.75mm, left=2mm]

In what sense is this ``too little'' innovation from a welfare
perspective? What policy instruments could close the gap?

\end{tcolorbox}
\end{frame}

\section{Innovation with Rivals and
Entry}\label{innovation-with-rivals-and-entry}

\begin{frame}{Incentives to innovate with rivals: intuition first}
\protect\phantomsection\label{incentives-to-innovate-with-rivals-intuition-first}
Innovation incentives are not monotone in ``competition intensity.''

\begin{itemize}
\tightlist
\item
  They depend on market structure (\(n\), differentiation, price vs
  quantity competition).
\item
  They depend on technology and institutions (protectability,
  spillovers).
\end{itemize}

\begin{tcolorbox}[enhanced jigsaw, opacitybacktitle=0.6, colbacktitle=quarto-callout-note-color!10!white, breakable, bottomrule=.15mm, rightrule=.15mm, colframe=quarto-callout-note-color-frame, colback=white, bottomtitle=1mm, toprule=.15mm, opacityback=0, title=\textcolor{quarto-callout-note-color}{\faInfo}\hspace{0.5em}{Rule of thumb}, titlerule=0mm, arc=.35mm, coltitle=black, toptitle=1mm, leftrule=.75mm, left=2mm]

More rivals reduce baseline profits (discouraging innovation), but can
raise the value of becoming the low-cost firm (encouraging innovation).

\end{tcolorbox}
\end{frame}

\begin{frame}{Number of rivals: Cournot intuition}
\protect\phantomsection\label{number-of-rivals-cournot-intuition}
In linear Cournot, R\&D incentives can follow an \textbf{inverse-U} as
\(n\) rises:

\begin{itemize}
\tightlist
\item
  \textbf{Competition effect:} more firms compress profits for everyone.
\item
  \textbf{Competitive advantage effect:} a cost lead is more valuable
  when many higher-cost rivals remain.
\end{itemize}

Which force dominates is an empirical question and can vary by industry.
\end{frame}

\begin{frame}{Entry threat framework: incumbent vs entrant}
\protect\phantomsection\label{entry-threat-framework-incumbent-vs-entrant}
Consider an incumbent monopolist facing a potential entrant:

\begin{itemize}
\tightlist
\item
  Innovation lowers marginal cost from \(c_0\) to \(c_1<c_0\).
\item
  The entrant can profitably enter only if it obtains the innovation.
\item
  Let \(\pi^d(c_i,c_j)\) be firm \(i\)'s duopoly profit when own cost is
  \(c_i\) and rival cost is \(c_j\).
\end{itemize}

Timing:

\begin{enumerate}
\tightlist
\item
  Firms compete for innovation (patent auction / R\&D race).
\item
  Innovation is allocated to the higher-valuation firm.
\item
  Entry decision and product-market competition occur after allocation.
\end{enumerate}
\end{frame}

\begin{frame}{Who values innovation more?}
\protect\phantomsection\label{who-values-innovation-more}
Payoff logic by winner:

\begin{itemize}
\tightlist
\item
  If the \textbf{incumbent} wins: market stays monopoly, payoff
  \(\pi^m(c_1)\).
\item
  If the \textbf{entrant} wins: entry occurs, payoffs become
  \(\pi^d(c_0,c_1)\) for the incumbent and \(\pi^d(c_1,c_0)\) for the
  entrant.
\end{itemize}

So per-period valuations are:

\[
V_I = \pi^m(c_1) - \pi^d(c_0, c_1), \qquad
V_E = \pi^d(c_1, c_0)
\]
\end{frame}

\begin{frame}{Entry threat: pre-emption condition (Gilbert and Newbery
1982)}
\protect\phantomsection\label{entry-threat-pre-emption-condition-gilbertpreemptivepatentingpersistence1982}
The incumbent has stronger innovation incentives when:

\[
V_I > V_E
\quad \Leftrightarrow \quad
\pi^m(c_1) > \pi^d(c_1, c_0) + \pi^d(c_0, c_1)
\]

This is more likely when products are close substitutes, so entry would
sharply reduce incumbent profits.

\begin{tcolorbox}[enhanced jigsaw, opacitybacktitle=0.6, colbacktitle=quarto-callout-important-color!10!white, breakable, bottomrule=.15mm, rightrule=.15mm, colframe=quarto-callout-important-color-frame, colback=white, bottomtitle=1mm, toprule=.15mm, opacityback=0, title=\textcolor{quarto-callout-important-color}{\faExclamation}\hspace{0.5em}{Interpretation}, titlerule=0mm, arc=.35mm, coltitle=black, toptitle=1mm, leftrule=.75mm, left=2mm]

For incumbents, innovation has a \textbf{dual payoff}: efficiency gain
(lower cost) plus market-structure protection (deterring entry). That is
the pre-emption channel.

\end{tcolorbox}
\end{frame}

\begin{frame}{Discussion: when does pre-emption fail?}
\protect\phantomsection\label{discussion-when-does-pre-emption-fail}
\begin{tcolorbox}[enhanced jigsaw, opacitybacktitle=0.6, colbacktitle=quarto-callout-note-color!10!white, breakable, bottomrule=.15mm, rightrule=.15mm, colframe=quarto-callout-note-color-frame, colback=white, bottomtitle=1mm, toprule=.15mm, opacityback=0, title=\textcolor{quarto-callout-note-color}{\faInfo}\hspace{0.5em}{Discussion question}, titlerule=0mm, arc=.35mm, coltitle=black, toptitle=1mm, leftrule=.75mm, left=2mm]

Under what market conditions might an entrant have stronger innovation
incentives than the incumbent? What features would reverse the
pre-emption result?

\end{tcolorbox}
\end{frame}

\begin{frame}{Answer: when entrant incentives dominate}
\protect\phantomsection\label{answer-when-entrant-incentives-dominate}
Entrant incentives are stronger when:

\[
V_E > V_I
\quad \Leftrightarrow \quad
\pi^d(c_1,c_0) + \pi^d(c_0,c_1) > \pi^m(c_1)
\]

\begin{itemize}
\tightlist
\item
  \textbf{Weak business-stealing effect}: products are more
  differentiated, so entry hurts the incumbent less.
\item
  \textbf{Low market-structure protection value}: moving from duopoly to
  monopoly is not worth much to the incumbent.
\item
  \textbf{High entrant upside from innovation}: entrant post-entry
  profits \(\pi^d(c_1,c_0)\) are large (e.g., efficient entry, strong
  demand segment).
\item
  \textbf{Large incumbent replacement effect}: the incumbent already has
  substantial rents without innovating, so incremental gains are
  limited.
\item
  \textbf{Weaker appropriability for incumbents} (strong
  spillovers/imitation): innovation is less effective as an
  entry-deterrence tool.
\end{itemize}

\begin{tcolorbox}[enhanced jigsaw, opacitybacktitle=0.6, colbacktitle=quarto-callout-important-color!10!white, breakable, bottomrule=.15mm, rightrule=.15mm, colframe=quarto-callout-important-color-frame, colback=white, bottomtitle=1mm, toprule=.15mm, opacityback=0, title=\textcolor{quarto-callout-important-color}{\faExclamation}\hspace{0.5em}{Bottom line}, titlerule=0mm, arc=.35mm, coltitle=black, toptitle=1mm, leftrule=.75mm, left=2mm]

Pre-emption fails when the incumbent's \textbf{entry-deterrence motive
is weak} and the entrant's \textbf{duopoly gain from innovating is
strong}.

\end{tcolorbox}
\end{frame}

\begin{frame}{Empirical example: strategic incentives and market share}
\protect\phantomsection\label{empirical-example-strategic-incentives-and-market-share}
\begin{tcolorbox}[enhanced jigsaw, opacitybacktitle=0.6, colbacktitle=quarto-callout-note-color!10!white, breakable, bottomrule=.15mm, rightrule=.15mm, colframe=quarto-callout-note-color-frame, colback=white, bottomtitle=1mm, toprule=.15mm, opacityback=0, title=\textcolor{quarto-callout-note-color}{\faInfo}\hspace{0.5em}{Pre-emptive innovation patterns (ReStud)}, titlerule=0mm, arc=.35mm, coltitle=black, toptitle=1mm, leftrule=.75mm, left=2mm]

Blundell et al. (1999) find that higher market share and market value
predict more patenting/innovations in UK manufacturing firms, consistent
with \textbf{strategic} incentives (including pre-emptive innovation) in
oligopolistic industries.

\end{tcolorbox}
\end{frame}

\section{Endogenous R\&D and Market Structure: Dasgupta and Stiglitz
(1980)}\label{endogenous-rd-and-market-structure-dasguptaindustrialstructurenature1980}

\begin{frame}{Big Question}
\protect\phantomsection\label{big-question}
In Dasgupta-Stiglitz, firms choose R\&D and entry is endogenous.\\
The central question is: \textbf{what jointly determines innovation
intensity and concentration?}

Key forces:

\begin{itemize}
\tightlist
\item
  \textbf{Demand conditions} (elasticity, market size)
\item
  \textbf{R\&D technology} (how effectively spending lowers cost)
\item
  \textbf{Free entry} (how many firms can cover R\&D outlays)
\end{itemize}
\end{frame}

\begin{frame}{Notation for this section}
\protect\phantomsection\label{notation-for-this-section}
{\def\LTcaptype{none} % do not increment counter
\begin{longtable}[]{@{}
  >{\raggedright\arraybackslash}p{(\linewidth - 4\tabcolsep) * \real{0.3478}}
  >{\raggedright\arraybackslash}p{(\linewidth - 4\tabcolsep) * \real{0.2609}}
  >{\raggedright\arraybackslash}p{(\linewidth - 4\tabcolsep) * \real{0.3913}}@{}}
\toprule\noalign{}
\begin{minipage}[b]{\linewidth}\raggedright
Symbol
\end{minipage} & \begin{minipage}[b]{\linewidth}\raggedright
Type
\end{minipage} & \begin{minipage}[b]{\linewidth}\raggedright
Meaning
\end{minipage} \\
\midrule\noalign{}
\endhead
\(x\) & Choice & R\&D expenditure per firm \\
\(c(x)\) & Function & Marginal cost, with \(c'(x)<0\) \\
\(N\) & Outcome & Number of active firms (endogenous) \\
\(\varepsilon\) & Parameter & Market demand elasticity \\
\(\alpha\) & Parameter & R\&D cost elasticity,
\(\alpha=-\frac{dc(x)}{dx}\frac{x}{c}\) \\
\bottomrule\noalign{}
\end{longtable}
}
\end{frame}

\begin{frame}{Planner benchmark: why scale matters}
\protect\phantomsection\label{planner-benchmark-why-scale-matters}
Planner problem:

\[
\max_{x,Q}\; V(x,Q)=U(Q)-c(x)Q-x
\]

FOC for R\&D:

\[
-c'(x)Q=1
\]

Interpretation: the marginal benefit of R\&D is proportional to output
scale \(Q\), so larger markets justify more cost-reducing innovation.
\end{frame}

\begin{frame}{Empirical anchor: market size and innovation}
\protect\phantomsection\label{empirical-anchor-market-size-and-innovation}
\begin{tcolorbox}[enhanced jigsaw, opacitybacktitle=0.6, colbacktitle=quarto-callout-note-color!10!white, breakable, bottomrule=.15mm, rightrule=.15mm, colframe=quarto-callout-note-color-frame, colback=white, bottomtitle=1mm, toprule=.15mm, opacityback=0, title=\textcolor{quarto-callout-note-color}{\faInfo}\hspace{0.5em}{Pharmaceutical innovation responds to demand}, titlerule=0mm, arc=.35mm, coltitle=black, toptitle=1mm, leftrule=.75mm, left=2mm]

Acemoglu and Linn (2004) show that larger potential markets lead to:

\begin{itemize}
\tightlist
\item
  more pharmaceutical innovation (new drugs / new molecular entities)
\item
  consistent evidence that innovation incentives rise with market scale
\end{itemize}

\end{tcolorbox}
\end{frame}

\begin{frame}{Industry side: firms choose output and R\&D}
\protect\phantomsection\label{industry-side-firms-choose-output-and-rd}
Assume symmetric Cournot competition. Firm \(i\) solves:

\[
\max_{x_i,q_i}\; \pi_i=[P(Q)-c(x_i)]q_i-x_i,\qquad Q=q_i+q_{-i}
\]

In symmetric equilibrium, \(q_i^*=Q^*/N^*\) and \(x_i^*=x^*\).
\end{frame}

\begin{frame}{Two equilibrium conditions}
\protect\phantomsection\label{two-equilibrium-conditions}
From Cournot pricing (perceived elasticity \(N\varepsilon\)):

\[
\frac{P-c}{P}=\frac{1}{N\varepsilon}
\quad\Leftrightarrow\quad
P-c=\frac{P}{N\varepsilon}
\]

From free entry (zero profit):

\[
(P-c)\frac{Q}{N}-x=0
\quad\Leftrightarrow\quad
P-c=\frac{Nx}{Q}
\]

Entry continues until operating margins exactly cover per-firm R\&D
spending.
\end{frame}

\begin{frame}{Deriving concentration in one step}
\protect\phantomsection\label{deriving-concentration-in-one-step}
Equate the two expressions for the margin:

\[
\frac{P}{N\varepsilon}=\frac{Nx}{Q}
\]

Use the R\&D elasticity definition with the R\&D FOC (\(-c'(x)q=1\)),
which implies:

\[
\frac{x}{q}=\alpha c
\quad\Rightarrow\quad
\frac{Nx}{Q}=\alpha c
\]

So:

\[
\frac{P}{N\varepsilon}=\alpha c
\]

Divide by \(P\) and substitute \(c/P=1-\frac{1}{N\varepsilon}\):

\[
\frac{1}{N\varepsilon}
=
\alpha\!\left(1-\frac{1}{N\varepsilon}\right)
\quad\Rightarrow\quad
\frac{1}{N^*}
=
\frac{1}{\varepsilon}\cdot\frac{\alpha}{1+\alpha}
\]
\end{frame}

\begin{frame}{Core result: equilibrium concentration}
\protect\phantomsection\label{core-result-equilibrium-concentration}
\[
\frac{1}{N^*}=\frac{1}{\varepsilon}\cdot\frac{\alpha}{1+\alpha}
\]

\begin{tcolorbox}[enhanced jigsaw, opacitybacktitle=0.6, colbacktitle=quarto-callout-note-color!10!white, breakable, bottomrule=.15mm, rightrule=.15mm, colframe=quarto-callout-note-color-frame, colback=white, bottomtitle=1mm, toprule=.15mm, opacityback=0, title=\textcolor{quarto-callout-note-color}{\faInfo}\hspace{0.5em}{Interpretation}, titlerule=0mm, arc=.35mm, coltitle=black, toptitle=1mm, leftrule=.75mm, left=2mm]

Concentration and innovation are jointly determined.\\
Higher R\&D effectiveness (\(\alpha\)) supports higher concentration,
while more elastic demand (\(\varepsilon\)) supports lower
concentration.

\end{tcolorbox}
\end{frame}

\begin{frame}{Numerical check}
\protect\phantomsection\label{numerical-check}
Using \$ \frac{1}{N^*}=\frac{1}{\varepsilon}\cdot\frac{\alpha}{1+\alpha}
\$:

\begin{itemize}
\tightlist
\item
  If \(\varepsilon=2\) and \(\alpha=1\), then
  \(\frac{1}{N^*}=\frac{1}{4}\) so \(N^*=4\).
\item
  If \(\varepsilon=2\) and \(\alpha=3\), then
  \(\frac{1}{N^*}=\frac{3}{8}\) so \(N^*\approx2.7\).
\end{itemize}

\begin{tcolorbox}[enhanced jigsaw, opacitybacktitle=0.6, colbacktitle=quarto-callout-note-color!10!white, breakable, bottomrule=.15mm, rightrule=.15mm, colframe=quarto-callout-note-color-frame, colback=white, bottomtitle=1mm, toprule=.15mm, opacityback=0, title=\textcolor{quarto-callout-note-color}{\faInfo}\hspace{0.5em}{Discussion question}, titlerule=0mm, arc=.35mm, coltitle=black, toptitle=1mm, leftrule=.75mm, left=2mm]

If policy raises R\&D effectiveness (\(\alpha\)), the model predicts
\textbf{more concentration} (lower \(N^*\)).\\
When is that welfare-improving once we account for both dynamic gains
and static mark-up losses?

\end{tcolorbox}
\end{frame}

\begin{frame}{Comparative statics and trade-offs}
\protect\phantomsection\label{comparative-statics-and-trade-offs}
With \(P(Q)=\sigma Q^{-\varepsilon}\) and \(c(x)=\beta x^{-\alpha}\):

\begin{itemize}
\tightlist
\item
  \(\frac{\partial(1/N^*)}{\partial\alpha}>0\): more effective R\&D
  tends to increase concentration.
\item
  \(\frac{\partial(1/N^*)}{\partial\varepsilon}<0\): more elastic demand
  tends to reduce concentration.
\item
  \(\frac{\partial x^*}{\partial \sigma}>0\): larger markets raise R\&D
  effort.
\item
  Typically, \(x^*(N+1)<x^*(N)\) but \(Q^*(N+1)>Q^*(N)\): more firms
  improve static efficiency yet can weaken per-firm innovation
  incentives.
\end{itemize}

This is the core dynamic-static tension in endogenous market-structure
models.
\end{frame}

\begin{frame}{Summary and next week}
\protect\phantomsection\label{summary-and-next-week}
\textbf{Summary}

\begin{itemize}
\tightlist
\item
  Innovation value depends on the objective: \textbf{\(\Delta \pi\)
  (private)} versus \textbf{\(\Delta W\) (social)}
\item
  Replacement effect: pre-innovation rents reduce the incumbent's
  incremental gain from innovation
\item
  With entry threat, innovation can be worth more because it changes
  \textbf{market structure} (monopoly vs duopoly)
\item
  In oligopoly, incentives reflect competing forces (competition effect
  vs competitive advantage effect)
\end{itemize}

\textbf{Next week: patents and IPRs}

\begin{itemize}
\tightlist
\item
  Patents as incentives: monopoly rights vs.~dynamic efficiency
\item
  Patent races and timing
\item
  Disclosure, licensing, and welfare
\item
  Horizontal and vertical innovation (brief)
\end{itemize}
\end{frame}

\begin{frame}{References}
\protect\phantomsection\label{references}
\protect\phantomsection\label{refs}
\begin{CSLReferences}{1}{1}
\bibitem[\citeproctext]{ref-acemogluMarketSizeInnovation2004}
Acemoglu, Daron, and Joshua Linn. 2004. {`Market Size in Innovation:
Theory and Evidence from the Pharmaceutical Industry'}. \emph{Quarterly
Journal of Economics} 119 (3): 1049--90.
\url{https://doi.org/10.1162/0033553041502144}.

\bibitem[\citeproctext]{ref-aghionCompetitionInnovationInvertedU2005}
Aghion, Philippe, Nick Bloom, Richard Blundell, Rachel Griffith, and
Peter Howitt. 2005. {`Competition and Innovation: An Inverted-u
Relationship'}. \emph{Quarterly Journal of Economics} 120 (2): 701--28.
\url{https://doi.org/10.1093/qje/120.2.701}.

\bibitem[\citeproctext]{ref-arrowEconomicWelfareAllocation1972}
Arrow, K. J. 1972. {`Economic Welfare and the Allocation of Resources
for Invention'}. In \emph{Readings in Industrial Economics: Volume Two:
Private Enterprise and State Intervention}, edited by Charles K. Rowley.
Macmillan Education UK.
\url{https://doi.org/10.1007/978-1-349-15486-9_13}.

\bibitem[\citeproctext]{ref-blundellMarketShareMarket1999}
Blundell, Richard, Rachel Griffith, and John Van Reenen. 1999. {`Market
Share, Market Value and Innovation in a Panel of British Manufacturing
Firms'}. \emph{The Review of Economic Studies} 66 (3): 529--54.
\url{https://doi.org/10.1111/1467-937X.00097}.

\bibitem[\citeproctext]{ref-dasguptaIndustrialStructureNature1980}
Dasgupta, Partha, and Joseph Stiglitz. 1980. {`Industrial Structure and
the Nature of Innovative Activity'}. \emph{The Economic Journal} 90
(358): 266--93. \url{https://doi.org/10.2307/2231788}.

\bibitem[\citeproctext]{ref-gilbertPreemptivePatentingPersistence1982}
Gilbert, Richard, and David M. Newbery. 1982. {`Preemptive Patenting and
the Persistence of Monopoly'}. \emph{American Economic Review} 72 (3):
514--26.

\bibitem[\citeproctext]{ref-trajtenbergWelfareAnalysisProduct1989}
Trajtenberg, Manuel. 1989. {`The Welfare Analysis of Product
Innovations, with an Application to Computed Tomography Scanners'}.
\emph{Journal of Political Economy} 97 (2): 444--79.
\url{https://doi.org/10.1086/261611}.

\end{CSLReferences}
\end{frame}




\end{document}
