% Options for packages loaded elsewhere
% Options for packages loaded elsewhere
\PassOptionsToPackage{unicode}{hyperref}
\PassOptionsToPackage{hyphens}{url}
%
\documentclass[
  ignorenonframetext,
  british,
]{beamer}
\newif\ifbibliography
\usepackage{pgfpages}
\setbeamertemplate{caption}[numbered]
\setbeamertemplate{caption label separator}{: }
\setbeamercolor{caption name}{fg=normal text.fg}
\beamertemplatenavigationsymbolsempty
% remove section numbering
\setbeamertemplate{part page}{
  \centering
  \begin{beamercolorbox}[sep=16pt,center]{part title}
    \usebeamerfont{part title}\insertpart\par
  \end{beamercolorbox}
}
\setbeamertemplate{section page}{
  \centering
  \begin{beamercolorbox}[sep=12pt,center]{section title}
    \usebeamerfont{section title}\insertsection\par
  \end{beamercolorbox}
}
\setbeamertemplate{subsection page}{
  \centering
  \begin{beamercolorbox}[sep=8pt,center]{subsection title}
    \usebeamerfont{subsection title}\insertsubsection\par
  \end{beamercolorbox}
}
% Prevent slide breaks in the middle of a paragraph
\widowpenalties 1 10000
\raggedbottom
\AtBeginPart{
  \frame{\partpage}
}
\AtBeginSection{
  \ifbibliography
  \else
    \frame{\sectionpage}
  \fi
}
\AtBeginSubsection{
  \frame{\subsectionpage}
}
\usepackage{iftex}
\ifPDFTeX
  \usepackage[T1]{fontenc}
  \usepackage[utf8]{inputenc}
  \usepackage{textcomp} % provide euro and other symbols
\else % if luatex or xetex
  \usepackage{unicode-math} % this also loads fontspec
  \defaultfontfeatures{Scale=MatchLowercase}
  \defaultfontfeatures[\rmfamily]{Ligatures=TeX,Scale=1}
\fi
\usepackage{lmodern}

\ifPDFTeX\else
  % xetex/luatex font selection
\fi
% Use upquote if available, for straight quotes in verbatim environments
\IfFileExists{upquote.sty}{\usepackage{upquote}}{}
\IfFileExists{microtype.sty}{% use microtype if available
  \usepackage[]{microtype}
  \UseMicrotypeSet[protrusion]{basicmath} % disable protrusion for tt fonts
}{}
\makeatletter
\@ifundefined{KOMAClassName}{% if non-KOMA class
  \IfFileExists{parskip.sty}{%
    \usepackage{parskip}
  }{% else
    \setlength{\parindent}{0pt}
    \setlength{\parskip}{6pt plus 2pt minus 1pt}}
}{% if KOMA class
  \KOMAoptions{parskip=half}}
\makeatother


\usepackage{longtable,booktabs,array}
\newcounter{none} % for unnumbered tables
\usepackage{calc} % for calculating minipage widths
\usepackage{caption}
% Make caption package work with longtable
\makeatletter
\def\fnum@table{\tablename~\thetable}
\makeatother
\usepackage{graphicx}
\makeatletter
\newsavebox\pandoc@box
\newcommand*\pandocbounded[1]{% scales image to fit in text height/width
  \sbox\pandoc@box{#1}%
  \Gscale@div\@tempa{\textheight}{\dimexpr\ht\pandoc@box+\dp\pandoc@box\relax}%
  \Gscale@div\@tempb{\linewidth}{\wd\pandoc@box}%
  \ifdim\@tempb\p@<\@tempa\p@\let\@tempa\@tempb\fi% select the smaller of both
  \ifdim\@tempa\p@<\p@\scalebox{\@tempa}{\usebox\pandoc@box}%
  \else\usebox{\pandoc@box}%
  \fi%
}
% Set default figure placement to htbp
\def\fps@figure{htbp}
\makeatother


% definitions for citeproc citations
\NewDocumentCommand\citeproctext{}{}
\NewDocumentCommand\citeproc{mm}{%
  \begingroup\def\citeproctext{#2}\cite{#1}\endgroup}
\makeatletter
 % allow citations to break across lines
 \let\@cite@ofmt\@firstofone
 % avoid brackets around text for \cite:
 \def\@biblabel#1{}
 \def\@cite#1#2{{#1\if@tempswa , #2\fi}}
\makeatother
\newlength{\cslhangindent}
\setlength{\cslhangindent}{1.5em}
\newlength{\csllabelwidth}
\setlength{\csllabelwidth}{3em}
\newenvironment{CSLReferences}[2] % #1 hanging-indent, #2 entry-spacing
 {\begin{list}{}{%
  \setlength{\itemindent}{0pt}
  \setlength{\leftmargin}{0pt}
  \setlength{\parsep}{0pt}
  % turn on hanging indent if param 1 is 1
  \ifodd #1
   \setlength{\leftmargin}{\cslhangindent}
   \setlength{\itemindent}{-1\cslhangindent}
  \fi
  % set entry spacing
  \setlength{\itemsep}{#2\baselineskip}}}
 {\end{list}}
\usepackage{calc}
\newcommand{\CSLBlock}[1]{\hfill\break\parbox[t]{\linewidth}{\strut\ignorespaces#1\strut}}
\newcommand{\CSLLeftMargin}[1]{\parbox[t]{\csllabelwidth}{\strut#1\strut}}
\newcommand{\CSLRightInline}[1]{\parbox[t]{\linewidth - \csllabelwidth}{\strut#1\strut}}
\newcommand{\CSLIndent}[1]{\hspace{\cslhangindent}#1}

\ifLuaTeX
\usepackage[bidi=basic,shorthands=off]{babel}
\else
\usepackage[bidi=default,shorthands=off]{babel}
\fi
\ifLuaTeX
  \usepackage{selnolig} % disable illegal ligatures
\fi


\setlength{\emergencystretch}{3em} % prevent overfull lines

\providecommand{\tightlist}{%
  \setlength{\itemsep}{0pt}\setlength{\parskip}{0pt}}



 


\usepackage{standalone}
\usepackage{tikz}
\usepackage{pgfplots}
\usetikzlibrary{decorations.pathreplacing}
\usetikzlibrary{arrows.meta}
\pgfplotsset{compat=1.18}
\makeatletter
\@ifpackageloaded{tcolorbox}{}{\usepackage[skins,breakable]{tcolorbox}}
\@ifpackageloaded{fontawesome5}{}{\usepackage{fontawesome5}}
\definecolor{quarto-callout-color}{HTML}{909090}
\definecolor{quarto-callout-note-color}{HTML}{0758E5}
\definecolor{quarto-callout-important-color}{HTML}{CC1914}
\definecolor{quarto-callout-warning-color}{HTML}{EB9113}
\definecolor{quarto-callout-tip-color}{HTML}{00A047}
\definecolor{quarto-callout-caution-color}{HTML}{FC5300}
\definecolor{quarto-callout-color-frame}{HTML}{acacac}
\definecolor{quarto-callout-note-color-frame}{HTML}{4582ec}
\definecolor{quarto-callout-important-color-frame}{HTML}{d9534f}
\definecolor{quarto-callout-warning-color-frame}{HTML}{f0ad4e}
\definecolor{quarto-callout-tip-color-frame}{HTML}{02b875}
\definecolor{quarto-callout-caution-color-frame}{HTML}{fd7e14}
\makeatother
\makeatletter
\@ifpackageloaded{caption}{}{\usepackage{caption}}
\AtBeginDocument{%
\ifdefined\contentsname
  \renewcommand*\contentsname{Table of contents}
\else
  \newcommand\contentsname{Table of contents}
\fi
\ifdefined\listfigurename
  \renewcommand*\listfigurename{List of Figures}
\else
  \newcommand\listfigurename{List of Figures}
\fi
\ifdefined\listtablename
  \renewcommand*\listtablename{List of Tables}
\else
  \newcommand\listtablename{List of Tables}
\fi
\ifdefined\figurename
  \renewcommand*\figurename{Figure}
\else
  \newcommand\figurename{Figure}
\fi
\ifdefined\tablename
  \renewcommand*\tablename{Table}
\else
  \newcommand\tablename{Table}
\fi
}
\@ifpackageloaded{float}{}{\usepackage{float}}
\floatstyle{ruled}
\@ifundefined{c@chapter}{\newfloat{codelisting}{h}{lop}}{\newfloat{codelisting}{h}{lop}[chapter]}
\floatname{codelisting}{Listing}
\newcommand*\listoflistings{\listof{codelisting}{List of Listings}}
\makeatother
\makeatletter
\makeatother
\makeatletter
\@ifpackageloaded{caption}{}{\usepackage{caption}}
\@ifpackageloaded{subcaption}{}{\usepackage{subcaption}}
\makeatother

\usepackage{bookmark}
\IfFileExists{xurl.sty}{\usepackage{xurl}}{} % add URL line breaks if available
\urlstyle{same}
\hypersetup{
  pdftitle={Lecture 3 - Innovation},
  pdfauthor={Gerhard Riener},
  pdflang={en-GB},
  hidelinks,
  pdfcreator={LaTeX via pandoc}}


\title{\texorpdfstring{Lecture 3 - Innovation}{Lecture 3 - Innovation}}
\author{Gerhard Riener}
\date{}

\begin{document}
\frame{\titlepage}


\section{Innovation and Market
Structure}\label{innovation-and-market-structure}

\begin{frame}{Innovation as a margin of competition}
\protect\phantomsection\label{innovation-as-a-margin-of-competition}
Innovation is the \textbf{dynamic} margin of competition: firms try to
change the game by shifting \textbf{costs} and \textbf{demand}.

\textbf{Why IO cares}

\begin{itemize}
\tightlist
\item
  Innovation changes \textbf{prices, mark-ups, and welfare} today
\item
  Innovation changes \textbf{market structure} tomorrow (entry/exit,
  dominance, concentration)
\item
  Policy trade-off: \textbf{static efficiency} (low prices) vs
  \textbf{dynamic efficiency} (strong incentives)
\end{itemize}
\end{frame}

\begin{frame}{Types of innovation (and what we model today)}
\protect\phantomsection\label{types-of-innovation-and-what-we-model-today}
\textbf{Process innovation} (this lecture's workhorse case)

\begin{itemize}
\tightlist
\item
  Lowers marginal (or average) cost, e.g.~a better production method
\item
  Think: a shift from \(c_0\) to \(c_1<c_0\)
\end{itemize}

\textbf{Product innovation}

\begin{itemize}
\tightlist
\item
  Raises willingness to pay, expands demand, or creates new varieties
\item
  Often analysed with differentiated products and quality ladders (later
  in the course)
\end{itemize}

\textbf{Stages (high level):} Research → Development → Adoption
\end{frame}

\begin{frame}{Learning objectives}
\protect\phantomsection\label{learning-objectives}
\begin{itemize}
\tightlist
\item
  Compute the \textbf{private value} of a process innovation under
  different market structures
\item
  Define \textbf{drastic vs non-drastic} innovations and explain the
  \textbf{replacement effect} (Arrow)
\item
  Understand why innovation can have \textbf{strategic value} when it
  affects \textbf{entry} and market structure
\item
  See how models with \textbf{free entry + endogenous R\&D} link
  innovation to \textbf{concentration} (Dasgupta--Stiglitz)
\end{itemize}

\begin{tcolorbox}[enhanced jigsaw, titlerule=0mm, leftrule=.75mm, toprule=.15mm, opacitybacktitle=0.6, colback=white, bottomtitle=1mm, coltitle=black, breakable, colframe=quarto-callout-note-color-frame, toptitle=1mm, title=\textcolor{quarto-callout-note-color}{\faInfo}\hspace{0.5em}{Today: roadmap}, arc=.35mm, rightrule=.15mm, bottomrule=.15mm, left=2mm, colbacktitle=quarto-callout-note-color!10!white, opacityback=0]

\begin{enumerate}
\tightlist
\item
  Benchmark model: value of a process innovation under
  \textbf{monopoly}, \textbf{perfect competition}, and the
  \textbf{social planner}
\item
  \textbf{Drastic vs non-drastic} innovations and the
  \textbf{replacement effect}
\item
  Innovation incentives in \textbf{oligopoly} and under \textbf{entry
  threat}
\item
  Seminal results on \textbf{concentration} and \textbf{R\&D}
  (Dasgupta--Stiglitz)
\end{enumerate}

\end{tcolorbox}
\end{frame}

\begin{frame}{Market structure and innovation: a two-way relationship}
\protect\phantomsection\label{market-structure-and-innovation-a-two-way-relationship}
\textbf{Market structure → innovation}

\begin{itemize}
\tightlist
\item
  Competition affects profits, appropriability, and the gain from
  becoming ``better'' than rivals
\end{itemize}

\textbf{Innovation → market structure}

\begin{itemize}
\tightlist
\item
  Cost/demand shifts affect entry, market shares, and concentration (and
  can create dominance)
\end{itemize}
\end{frame}

\begin{frame}{Measuring incentives: willingness to pay (WTP)}
\protect\phantomsection\label{measuring-incentives-willingness-to-pay-wtp}
\begin{itemize}
\tightlist
\item
  \textbf{Firm WTP}: the max lump-sum payment that leaves profits
  unchanged \(\Rightarrow \ \text{WTP} = \Delta \pi\)
\item
  \textbf{Planner WTP}: the max lump-sum payment that leaves welfare
  unchanged \(\Rightarrow \ \text{WTP} = \Delta W\)
\end{itemize}

In what follows we compute WTP for: \textbf{Monopoly} (before/after),
\textbf{perfect competition} (before; exclusive rights after), and the
\textbf{social planner}.
\end{frame}

\section{Benchmarks: Value of a Process
Innovation}\label{benchmarks-value-of-a-process-innovation}

\begin{frame}{Setup: linear demand + cost-reducing innovation}
\protect\phantomsection\label{setup-linear-demand-cost-reducing-innovation}
\textbf{Inverse demand}

\[
P(Q) = A - Q
\]

\textbf{Technology}

\begin{itemize}
\tightlist
\item
  Constant marginal cost \(c \in \{c_0, c_1\}\)
\item
  Process innovation reduces marginal cost from \(c_0\) to \(c_1\) with
  \(c_1 < c_0 < A\)
\end{itemize}

\begin{tcolorbox}[enhanced jigsaw, titlerule=0mm, leftrule=.75mm, toprule=.15mm, opacitybacktitle=0.6, colback=white, bottomtitle=1mm, coltitle=black, breakable, colframe=quarto-callout-note-color-frame, toptitle=1mm, title=\textcolor{quarto-callout-note-color}{\faInfo}\hspace{0.5em}{Goal}, arc=.35mm, rightrule=.15mm, bottomrule=.15mm, left=2mm, colbacktitle=quarto-callout-note-color!10!white, opacityback=0]

Compute the value of moving from \(c_0\) to \(c_1\) under different
market structures.

\end{tcolorbox}
\end{frame}

\begin{frame}{Monopoly benchmark: problem and solution}
\protect\phantomsection\label{monopoly-benchmark-problem-and-solution}
\textbf{Problem}

\[
\max_Q \ (A - Q - c)Q
\]

\textbf{Solution}

\[
Q^m(c) = \frac{A-c}{2}, \qquad P^m(c) = \frac{A+c}{2}, \qquad \pi^m(c) = \frac{(A-c)^2}{4}
\]
\end{frame}

\begin{frame}{Monopoly WTP for innovation}
\protect\phantomsection\label{monopoly-wtp-for-innovation}
\textbf{WTP for the innovation}

\[
\Delta \pi^m = \pi^m(c_1) - \pi^m(c_0) = \frac{(A-c_1)^2 - (A-c_0)^2}{4}
\]

\begin{tcolorbox}[enhanced jigsaw, titlerule=0mm, leftrule=.75mm, toprule=.15mm, opacitybacktitle=0.6, colback=white, bottomtitle=1mm, coltitle=black, breakable, colframe=quarto-callout-note-color-frame, toptitle=1mm, title=\textcolor{quarto-callout-note-color}{\faInfo}\hspace{0.5em}{Interpretation}, arc=.35mm, rightrule=.15mm, bottomrule=.15mm, left=2mm, colbacktitle=quarto-callout-note-color!10!white, opacityback=0]

\(\Delta \pi^m\) is the monopolist's value of \emph{exclusive} access to
the lower cost \(c_1\) (it is incremental because the firm already earns
rents at \(c_0\)).

\end{tcolorbox}
\end{frame}

\begin{frame}{Perfect competition benchmark (innovation creates rents)}
\protect\phantomsection\label{perfect-competition-benchmark-innovation-creates-rents}
\textbf{Before innovation (all firms at MC \(c_0\)):}

\begin{itemize}
\tightlist
\item
  Competitive price: \(P^{pc}_0 = c_0\)
\item
  Firm profit: \(\pi = 0\)
\end{itemize}

\textbf{After innovation (innovator has exclusive use / patent):}

\begin{itemize}
\tightlist
\item
  \textbf{Drastic}: innovator behaves as a monopolist with cost \(c_1\)
\item
  \textbf{Non-drastic}: innovator is constrained by the competitive
  fringe at cost \(c_0\) (limit pricing)
\end{itemize}

\textbf{Innovator profit / WTP}

\[
\Delta \pi^{pc} =
\begin{cases}
\pi^m(c_1) & \text{(drastic)}\\
(c_0 - c_1)(A - c_0) & \text{(non-drastic, } p=c_0 \text{)}
\end{cases}
\]

\textbf{Why the non-drastic formula?} If the innovator sets \(p=c_0\),
quantity is \(Q=A-c_0\), so profits are \((p-c_1)Q=(c_0-c_1)(A-c_0)\).
\end{frame}

\begin{frame}{Drastic innovation condition}
\protect\phantomsection\label{drastic-innovation-condition}
\[
P^m(c_1) < c_0
\quad \Leftrightarrow \quad
\frac{A+c_1}{2} < c_0
\quad \Leftrightarrow \quad
A + c_1 < 2c_0
\]

\resizebox{0.85\textwidth}{!}{\includestandalone{../figs/source/fig-innovation-drastic-threshold}}
\end{frame}

\begin{frame}{Numeric check: drastic or non-drastic?}
\protect\phantomsection\label{numeric-check-drastic-or-non-drastic}
Take \(A=20\), \(c_0=10\), \(c_1=6\).

\begin{itemize}
\tightlist
\item
  \textbf{Drastic condition}: \(A + c_1 = 26 > 2c_0 = 20\) →
  \textbf{non-drastic}
\item
  Innovator limit-prices at \(p = c_0 = 10\), sells \(Q = A - c_0 = 10\)
\item
  Competitive innovator profit:
  \((c_0 - c_1)(A - c_0) = 4 \times 10 = 40\)
\item
  Recall monopoly WTP:
  \(\Delta\pi^m = \frac{(14)^2 - (10)^2}{4} = \frac{196 - 100}{4} = 24\)
\end{itemize}

\begin{tcolorbox}[enhanced jigsaw, titlerule=0mm, leftrule=.75mm, toprule=.15mm, opacitybacktitle=0.6, colback=white, bottomtitle=1mm, coltitle=black, breakable, colframe=quarto-callout-note-color-frame, toptitle=1mm, title=\textcolor{quarto-callout-note-color}{\faInfo}\hspace{0.5em}{Replacement effect preview}, arc=.35mm, rightrule=.15mm, bottomrule=.15mm, left=2mm, colbacktitle=quarto-callout-note-color!10!white, opacityback=0]

The competitive innovator earns \(40 > 24\) (the monopolist's WTP). The
monopolist gains less because it already earns \(\pi^m(c_0) = 25\)
before innovating --- it is partly \textbf{replacing} existing rents.

\end{tcolorbox}
\end{frame}

\begin{frame}{Replacement effect (Arrow 1972)}
\protect\phantomsection\label{replacement-effect-arroweconomicwelfareallocation1972}
\begin{tcolorbox}[enhanced jigsaw, titlerule=0mm, leftrule=.75mm, toprule=.15mm, opacitybacktitle=0.6, colback=white, bottomtitle=1mm, coltitle=black, breakable, colframe=quarto-callout-note-color-frame, toptitle=1mm, title=\textcolor{quarto-callout-note-color}{\faInfo}\hspace{0.5em}{Key idea (replacement effect)}, arc=.35mm, rightrule=.15mm, bottomrule=.15mm, left=2mm, colbacktitle=quarto-callout-note-color!10!white, opacityback=0]

Under perfect competition, pre-innovation profits are zero, so
innovation mainly \textbf{creates} rents. Under monopoly, innovation
partly \textbf{replaces} existing rents.

\end{tcolorbox}
\end{frame}

\begin{frame}{Appropriability and imitation}
\protect\phantomsection\label{appropriability-and-imitation}
\emph{If the innovation cannot be protected (instant diffusion / perfect
imitation), competition drives price to \(c_1\) and the innovator's WTP
is zero. Most IO policy questions start from the fact that
appropriability is imperfect but positive (patents, secrecy, lead
time).}
\end{frame}

\begin{frame}{Replacement effect: a one-line comparison (drastic case)}
\protect\phantomsection\label{replacement-effect-a-one-line-comparison-drastic-case}
If the innovation is drastic and grants exclusive rights:

\[
\Delta \pi^{pc} = \pi^m(c_1), \qquad \Delta \pi^m = \pi^m(c_1) - \pi^m(c_0)
\]

\pause

\[
\Delta \pi^{pc} - \Delta \pi^m = \pi^m(c_0) > 0
\]

\begin{tcolorbox}[enhanced jigsaw, titlerule=0mm, leftrule=.75mm, toprule=.15mm, opacitybacktitle=0.6, colback=white, bottomtitle=1mm, coltitle=black, breakable, colframe=quarto-callout-important-color-frame, toptitle=1mm, title=\textcolor{quarto-callout-important-color}{\faExclamation}\hspace{0.5em}{Takeaway}, arc=.35mm, rightrule=.15mm, bottomrule=.15mm, left=2mm, colbacktitle=quarto-callout-important-color!10!white, opacityback=0]

Holding everything else fixed, an incumbent monopolist has
\textbf{weaker} incentives to innovate than a competitive industry
because it is ``replacing'' its own pre-innovation rents.

\end{tcolorbox}

\resizebox{0.95\textwidth}{!}{\includestandalone{../figs/source/fig-innovation-replacement-effect}}
\end{frame}

\begin{frame}{Empirical example: competition and innovation}
\protect\phantomsection\label{empirical-example-competition-and-innovation}
\begin{tcolorbox}[enhanced jigsaw, titlerule=0mm, leftrule=.75mm, toprule=.15mm, opacitybacktitle=0.6, colback=white, bottomtitle=1mm, coltitle=black, breakable, colframe=quarto-callout-note-color-frame, toptitle=1mm, title=\textcolor{quarto-callout-note-color}{\faInfo}\hspace{0.5em}{Inverted-U evidence (QJE)}, arc=.35mm, rightrule=.15mm, bottomrule=.15mm, left=2mm, colbacktitle=quarto-callout-note-color!10!white, opacityback=0]

(Aghion et al. 2005) document an \textbf{inverted-U} relationship
between product-market competition and innovation (patents/citations) in
UK firms, consistent with ``escape competition'' incentives for
neck-and-neck firms and a \textbf{replacement effect} for laggards.

\end{tcolorbox}

\resizebox{0.8\textwidth}{!}{\includestandalone{../figs/source/fig-innovation-inverted-u}}
\end{frame}

\begin{frame}{Social planner: value of a cost reduction}
\protect\phantomsection\label{social-planner-value-of-a-cost-reduction}
\textbf{Efficient output} (price equals marginal cost)

\[
Q^{sp}(c) = A - c
\]

\textbf{Total surplus (welfare):}

\[
W(c) = \int_0^{A-c} (A - Q - c) \, dQ = \frac{(A-c)^2}{2}
\]

\begin{tcolorbox}[enhanced jigsaw, titlerule=0mm, leftrule=.75mm, toprule=.15mm, opacitybacktitle=0.6, colback=white, bottomtitle=1mm, coltitle=black, breakable, colframe=quarto-callout-note-color-frame, toptitle=1mm, title=\textcolor{quarto-callout-note-color}{\faInfo}\hspace{0.5em}{Interpretation}, arc=.35mm, rightrule=.15mm, bottomrule=.15mm, left=2mm, colbacktitle=quarto-callout-note-color!10!white, opacityback=0]

In this linear example, \(\Delta W = 2\,\Delta \pi^m\): the planner
values the output expansion that the monopolist does not internalise.

\end{tcolorbox}
\end{frame}

\begin{frame}{Social planner WTP for innovation}
\protect\phantomsection\label{social-planner-wtp-for-innovation}
\textbf{Max WTP for innovation (social planner):}

\[
\Delta W = W(c_1) - W(c_0) = \frac{(A-c_1)^2 - (A-c_0)^2}{2}
\]
\end{frame}

\begin{frame}{Putting the benchmarks side-by-side}
\protect\phantomsection\label{putting-the-benchmarks-side-by-side}
{\def\LTcaptype{none} % do not increment counter
\begin{longtable}[]{@{}
  >{\raggedright\arraybackslash}p{(\linewidth - 4\tabcolsep) * \real{0.3333}}
  >{\raggedright\arraybackslash}p{(\linewidth - 4\tabcolsep) * \real{0.3333}}
  >{\raggedright\arraybackslash}p{(\linewidth - 4\tabcolsep) * \real{0.3333}}@{}}
\toprule\noalign{}
\begin{minipage}[b]{\linewidth}\raggedright
Environment
\end{minipage} & \begin{minipage}[b]{\linewidth}\raggedright
What the innovator captures
\end{minipage} & \begin{minipage}[b]{\linewidth}\raggedright
WTP / value
\end{minipage} \\
\midrule\noalign{}
\endhead
Monopoly (incumbent) & Incremental rents &
\(\Delta\pi^m=\pi^m(c_1)-\pi^m(c_0)\) \\
Perfect competition (pre) + exclusive rights (post) & Mostly created
rents & \(\Delta\pi^{pc}=\pi^m(c_1)\) if drastic; otherwise limit
pricing profit \\
Social planner & Full surplus gain & \(\Delta W=W(c_1)-W(c_0)\) \\
\bottomrule\noalign{}
\end{longtable}
}

\begin{tcolorbox}[enhanced jigsaw, titlerule=0mm, leftrule=.75mm, toprule=.15mm, opacitybacktitle=0.6, colback=white, bottomtitle=1mm, coltitle=black, breakable, colframe=quarto-callout-important-color-frame, toptitle=1mm, title=\textcolor{quarto-callout-important-color}{\faExclamation}\hspace{0.5em}{Policy interpretation}, arc=.35mm, rightrule=.15mm, bottomrule=.15mm, left=2mm, colbacktitle=quarto-callout-important-color!10!white, opacityback=0]

Private incentives and social value need not align: \textbf{\(\Delta W\)
can exceed \(\Delta\pi\)}, but market power created by IPRs also
generates static distortions. This is the basic patents trade-off.

\end{tcolorbox}
\end{frame}

\begin{frame}{Empirical example: private vs social returns}
\protect\phantomsection\label{empirical-example-private-vs-social-returns}
\begin{tcolorbox}[enhanced jigsaw, titlerule=0mm, leftrule=.75mm, toprule=.15mm, opacitybacktitle=0.6, colback=white, bottomtitle=1mm, coltitle=black, breakable, colframe=quarto-callout-note-color-frame, toptitle=1mm, title=\textcolor{quarto-callout-note-color}{\faInfo}\hspace{0.5em}{Welfare gains from product innovation (JPE)}, arc=.35mm, rightrule=.15mm, bottomrule=.15mm, left=2mm, colbacktitle=quarto-callout-note-color!10!white, opacityback=0]

Trajtenberg (1989) estimates consumer and producer surplus gains from
quality improvements in computed tomography (CT) scanners, illustrating
that the \textbf{social value} of innovation can greatly exceed the
innovator's private returns. (Trajtenberg 1989)

\end{tcolorbox}
\end{frame}

\begin{frame}{Quick numeric check (optional)}
\protect\phantomsection\label{quick-numeric-check-optional}
Take \(A=20\), \(c_0=10\), \(c_1=6\).

\begin{itemize}
\tightlist
\item
  Monopoly: \(\pi^m(c_0)=25\), \(\pi^m(c_1)=49\), so \(\Delta\pi^m=24\)
\item
  Planner: \(W(c_0)=50\), \(W(c_1)=98\), so \(\Delta W=48\) (twice the
  monopoly gain)
\end{itemize}

\begin{tcolorbox}[enhanced jigsaw, titlerule=0mm, leftrule=.75mm, toprule=.15mm, opacitybacktitle=0.6, colback=white, bottomtitle=1mm, coltitle=black, breakable, colframe=quarto-callout-note-color-frame, toptitle=1mm, title=\textcolor{quarto-callout-note-color}{\faInfo}\hspace{0.5em}{Discussion question}, arc=.35mm, rightrule=.15mm, bottomrule=.15mm, left=2mm, colbacktitle=quarto-callout-note-color!10!white, opacityback=0]

In what sense is this ``too little'' innovation from a welfare
perspective? What policy instruments could close the gap?

\end{tcolorbox}
\end{frame}

\section{Innovation with Rivals and
Entry}\label{innovation-with-rivals-and-entry}

\begin{frame}{Incentives to innovate in oligopoly}
\protect\phantomsection\label{incentives-to-innovate-in-oligopoly}
\textbf{Takeaway}

\begin{itemize}
\tightlist
\item
  Innovation incentives need not be monotone in ``competition
  intensity''
\item
  They depend on:

  \begin{itemize}
  \tightlist
  \item
    Number of firms (\(n\))
  \item
    Product differentiation
  \item
    Mode of competition (price vs quantity)
  \item
    Whether innovation is \textbf{protectable} (patents/lead time) and
    whether there are \textbf{spillovers}
  \end{itemize}
\end{itemize}

\begin{tcolorbox}[enhanced jigsaw, titlerule=0mm, leftrule=.75mm, toprule=.15mm, opacitybacktitle=0.6, colback=white, bottomtitle=1mm, coltitle=black, breakable, colframe=quarto-callout-note-color-frame, toptitle=1mm, title=\textcolor{quarto-callout-note-color}{\faInfo}\hspace{0.5em}{Rule of thumb}, arc=.35mm, rightrule=.15mm, bottomrule=.15mm, left=2mm, colbacktitle=quarto-callout-note-color!10!white, opacityback=0]

More rivals reduces baseline profits (hurting incentives), but can
increase the value of being the low-cost firm (helping incentives).

\end{tcolorbox}
\end{frame}

\begin{frame}{Impact of number of firms (Cournot intuition)}
\protect\phantomsection\label{impact-of-number-of-firms-cournot-intuition}
\textbf{Linear Cournot with \(n\) firms}: incentives can follow an
\textbf{inverse-U} as \(n\) increases.

\textbf{Two opposing effects as \(n \uparrow\):}

\begin{itemize}
\tightlist
\item
  \textbf{Competition effect}: profits for all firms fall (discourages
  innovation)
\item
  \textbf{Competitive advantage effect}: cost advantage becomes more
  valuable when there are more inefficient rivals (encourages
  innovation)
\end{itemize}
\end{frame}

\begin{frame}{Monopoly threatened by entry: setup}
\protect\phantomsection\label{monopoly-threatened-by-entry-setup}
\textbf{Incumbent monopoly faces potential entrant:}

\begin{itemize}
\tightlist
\item
  Both can acquire innovation reducing MC from \(c_0\) to \(c_1 < c_0\)
\item
  Entrant can enter profitably \textbf{only if} it gets innovation
\item
  Let \(\pi^d(c_i,c_j)\) denote firm \(i\)'s profit in the post-entry
  \textbf{duopoly} when own cost is \(c_i\) and the rival's cost is
  \(c_j\)
\end{itemize}

\textbf{Payoffs:}

\begin{itemize}
\tightlist
\item
  \textbf{If incumbent gets innovation}: entrant stays out

  \begin{itemize}
  \tightlist
  \item
    Incumbent: \(\pi^m(c_1)\); Entrant: \(0\)
  \end{itemize}
\item
  \textbf{If entrant gets innovation}: entrant enters

  \begin{itemize}
  \tightlist
  \item
    Incumbent: \(\pi^d(c_0, c_1)\); Entrant: \(\pi^d(c_1, c_0)\)
  \end{itemize}
\end{itemize}

\begin{tcolorbox}[enhanced jigsaw, titlerule=0mm, leftrule=.75mm, toprule=.15mm, opacitybacktitle=0.6, colback=white, bottomtitle=1mm, coltitle=black, breakable, colframe=quarto-callout-note-color-frame, toptitle=1mm, title=\textcolor{quarto-callout-note-color}{\faInfo}\hspace{0.5em}{Strategic value}, arc=.35mm, rightrule=.15mm, bottomrule=.15mm, left=2mm, colbacktitle=quarto-callout-note-color!10!white, opacityback=0]

Innovation can change market structure (monopoly vs duopoly), so
incentives are not only about production efficiency.

\end{tcolorbox}
\end{frame}

\begin{frame}{Entry threat: pre-emption incentives (Gilbert and Newbery
1982)}
\protect\phantomsection\label{entry-threat-pre-emption-incentives-gilbertpreemptivepatentingpersistence1982}
\textbf{Per-period value of innovation}

\[
V_I = \pi^m(c_1) - \pi^d(c_0, c_1)
\]

\[
V_E = \pi^d(c_1, c_0)
\]

\textbf{Incumbent has stronger incentives if}

\pause

\[
V_I > V_E \quad \Leftrightarrow \quad \pi^m(c_1) > \pi^d(c_1, c_0) + \pi^d(c_0, c_1)
\]

Typically when products are close substitutes (entry is very harmful).

\begin{tcolorbox}[enhanced jigsaw, titlerule=0mm, leftrule=.75mm, toprule=.15mm, opacitybacktitle=0.6, colback=white, bottomtitle=1mm, coltitle=black, breakable, colframe=quarto-callout-important-color-frame, toptitle=1mm, title=\textcolor{quarto-callout-important-color}{\faExclamation}\hspace{0.5em}{Interpretation}, arc=.35mm, rightrule=.15mm, bottomrule=.15mm, left=2mm, colbacktitle=quarto-callout-important-color!10!white, opacityback=0]

Here the incumbent may innovate \textbf{more} than an entrant because
innovation does two things at once: it lowers cost \emph{and} it keeps
the market in the monopoly regime (entry deterrence / pre-emption).

\end{tcolorbox}
\end{frame}

\begin{frame}{Discussion: when does pre-emption fail?}
\protect\phantomsection\label{discussion-when-does-pre-emption-fail}
Under what market conditions might an entrant have stronger innovation
incentives than the incumbent? What features would reverse the
pre-emption result?
\end{frame}

\begin{frame}{Empirical example: strategic incentives and market share}
\protect\phantomsection\label{empirical-example-strategic-incentives-and-market-share}
\begin{tcolorbox}[enhanced jigsaw, titlerule=0mm, leftrule=.75mm, toprule=.15mm, opacitybacktitle=0.6, colback=white, bottomtitle=1mm, coltitle=black, breakable, colframe=quarto-callout-note-color-frame, toptitle=1mm, title=\textcolor{quarto-callout-note-color}{\faInfo}\hspace{0.5em}{Pre-emptive innovation patterns (ReStud)}, arc=.35mm, rightrule=.15mm, bottomrule=.15mm, left=2mm, colbacktitle=quarto-callout-note-color!10!white, opacityback=0]

Blundell, Griffith, and Van Reenen (1999) find that higher market share
and market value predict more patenting/innovations in UK manufacturing
firms, consistent with \textbf{strategic} incentives (including
pre-emptive innovation) in oligopolistic industries. (Blundell et al.
1999)

\end{tcolorbox}
\end{frame}

\section{Seminal Contributions}\label{seminal-contributions}

\begin{frame}{Dasgupta--Stiglitz: endogenous R\&D and market structure
(Dasgupta and Stiglitz 1980)}
\protect\phantomsection\label{dasguptastiglitz-endogenous-rd-and-market-structure-dasguptaindustrialstructurenature1980}
They model R\&D as an \textbf{endogenous sunk cost}: firms spend on R\&D
to reduce marginal cost, and entry adjusts endogenously.

The key result is a clean link between:

\begin{itemize}
\tightlist
\item
  \textbf{Demand conditions} (elasticity, market size)
\item
  \textbf{Innovative opportunities} (how effective R\&D is at lowering
  costs)
\item
  \textbf{Equilibrium concentration} (how many firms survive under free
  entry)
\end{itemize}

The linear model gave us clean closed-form comparisons. The D-S
framework sacrifices tractability for generality --- but delivers a
structural result linking innovation to concentration.

\begin{tcolorbox}[enhanced jigsaw, titlerule=0mm, leftrule=.75mm, toprule=.15mm, opacitybacktitle=0.6, colback=white, bottomtitle=1mm, coltitle=black, breakable, colframe=quarto-callout-note-color-frame, toptitle=1mm, title=\textcolor{quarto-callout-note-color}{\faInfo}\hspace{0.5em}{Notation for this section}, arc=.35mm, rightrule=.15mm, bottomrule=.15mm, left=2mm, colbacktitle=quarto-callout-note-color!10!white, opacityback=0]

{\def\LTcaptype{none} % do not increment counter
\begin{longtable}[]{@{}lll@{}}
\toprule\noalign{}
Symbol & Type & Meaning \\
\midrule\noalign{}
\endhead
\(x\) & Choice & R\&D expenditure per firm \\
\(c(x)\) & Function & Marginal cost (decreasing in \(x\)) \\
\(N\) & Outcome & Number of active firms (endogenous) \\
\(\varepsilon\) & Parameter & Demand elasticity \\
\(\alpha\) & Parameter & Innovative potential (R\&D cost elasticity) \\
\bottomrule\noalign{}
\end{longtable}
}

\end{tcolorbox}
\end{frame}

\begin{frame}{Dasgupta--Stiglitz (1980): planner benchmark (Dasgupta and
Stiglitz 1980)}
\protect\phantomsection\label{dasguptastiglitz-1980-planner-benchmark-dasguptaindustrialstructurenature1980}
\textbf{Question}: Do more firms imply more innovation?

\textbf{Primitives:}

\begin{itemize}
\tightlist
\item
  \(U(Q)\) = gross social benefit
\item
  \(c = c(x)\) = constant marginal cost depending on R\&D expenditure
  \(x\)
\item
  Assumption: \(c'(x) < 0\) (R\&D reduces cost)
\end{itemize}

\textbf{Social optimum:}

\[
\max_{x,Q} \ V(x,Q) = U(Q) - c(x)Q - x
\]

\textbf{First-order condition in \(x\):}

\[
-c'(x)Q = 1
\]
\end{frame}

\begin{frame}{Planner benchmark: interpretation}
\protect\phantomsection\label{planner-benchmark-interpretation}
\begin{itemize}
\tightlist
\item
  Optimal R\&D increases with scale \(Q\): a larger market makes cost
  reduction more valuable
\end{itemize}
\end{frame}

\begin{frame}{Empirical example: market size and innovation}
\protect\phantomsection\label{empirical-example-market-size-and-innovation}
\begin{tcolorbox}[enhanced jigsaw, titlerule=0mm, leftrule=.75mm, toprule=.15mm, opacitybacktitle=0.6, colback=white, bottomtitle=1mm, coltitle=black, breakable, colframe=quarto-callout-note-color-frame, toptitle=1mm, title=\textcolor{quarto-callout-note-color}{\faInfo}\hspace{0.5em}{Pharmaceutical innovation responds to demand (QJE)}, arc=.35mm, rightrule=.15mm, bottomrule=.15mm, left=2mm, colbacktitle=quarto-callout-note-color!10!white, opacityback=0]

Acemoglu and Linn (2004) show that larger potential markets lead to more
pharmaceutical innovation (new drugs/new molecular entities), consistent
with models where the return to innovation rises with \textbf{scale}.
(Acemoglu and Linn 2004)

\end{tcolorbox}
\end{frame}

\begin{frame}{Dasgupta--Stiglitz (1980): firm's problem (Dasgupta and
Stiglitz 1980)}
\protect\phantomsection\label{dasguptastiglitz-1980-firms-problem-dasguptaindustrialstructurenature1980}
\textbf{Firm \(i\)'s problem (symmetric firms):}

\[
\max_{x_i, q_i} \ \pi(x_i, q_i) = [P(q_i + q_{-i}) - c(x_i)]q_i - x_i
\]

\textbf{Symmetric equilibrium:}

\begin{itemize}
\tightlist
\item
  \(q_1^* = \cdots = q_N^* = Q^*/N\) and
  \(x_1^* = \cdots = x_N^* = x^*\)
\end{itemize}
\end{frame}

\begin{frame}{Dasgupta--Stiglitz (1980): equilibrium pricing (Dasgupta
and Stiglitz 1980)}
\protect\phantomsection\label{dasguptastiglitz-1980-equilibrium-pricing-dasguptaindustrialstructurenature1980}
\textbf{First-order condition in \(q_i\):}

\[
P(Q^*)\left(1 - \frac{1}{N\varepsilon}\right) = c(x^*)
\]

where \(\varepsilon = -\frac{\partial Q}{\partial P}\frac{P}{Q}\) is the
demand elasticity.

\textbf{Interpretation:}

\begin{itemize}
\tightlist
\item
  The term \(\left(1-\frac{1}{N\varepsilon}\right)\) captures market
  power via the perceived demand elasticity under Cournot

  \begin{itemize}
  \tightlist
  \item
    More firms (\(N\uparrow\)) → perceived demand becomes more elastic →
    mark-ups fall
  \end{itemize}
\end{itemize}
\end{frame}

\begin{frame}{Free entry condition (Dasgupta and Stiglitz 1980)}
\protect\phantomsection\label{free-entry-condition-dasguptaindustrialstructurenature1980}
\textbf{Zero profit condition} (free entry):

\[
\pi = 0 \quad \Rightarrow \quad (P - c)\frac{Q^*}{N^*} - x^* = 0
\]

\textbf{Rearrange:}

\[
P - c = N^* \frac{x^*}{Q^*}
\]
\end{frame}

\begin{frame}{Deriving the concentration formula}
\protect\phantomsection\label{deriving-the-concentration-formula}
\textbf{Combine the two conditions:}

\begin{itemize}
\tightlist
\item
  From pricing FOC: \(P - c = \frac{P}{N\varepsilon}\)
\item
  From free entry: \(P - c = \frac{Nx}{Q}\)
\end{itemize}

\textbf{Equating and using the R\&D FOC} (\(-c'(x)q = 1\), so
\(x/(cq) = \alpha\)):

\[
\frac{P}{N\varepsilon} = \frac{Nx}{Q} = \alpha c
\]

\textbf{Dividing by \(P\) and noting} \(c/P = 1 - 1/(N\varepsilon)\)
from the pricing FOC:

\[
\frac{1}{N\varepsilon} = \alpha\!\left(1 - \frac{1}{N\varepsilon}\right) \quad \Rightarrow \quad \frac{1}{N^*} = \frac{1}{\varepsilon}\cdot\frac{\alpha}{1+\alpha}
\]
\end{frame}

\begin{frame}{Equilibrium concentration (Dasgupta and Stiglitz 1980)}
\protect\phantomsection\label{equilibrium-concentration-dasguptaindustrialstructurenature1980}
\textbf{Define innovative potential:}

\[
\alpha = -\frac{dc(x)}{dx}\frac{x}{c}
\]

\textbf{Equilibrium concentration satisfies:}

\pause

\[
\frac{1}{N^*} = \frac{1}{\varepsilon} \cdot \frac{\alpha}{1+\alpha}
\]

\begin{tcolorbox}[enhanced jigsaw, titlerule=0mm, leftrule=.75mm, toprule=.15mm, opacitybacktitle=0.6, colback=white, bottomtitle=1mm, coltitle=black, breakable, colframe=quarto-callout-note-color-frame, toptitle=1mm, title=\textcolor{quarto-callout-note-color}{\faInfo}\hspace{0.5em}{Interpretation}, arc=.35mm, rightrule=.15mm, bottomrule=.15mm, left=2mm, colbacktitle=quarto-callout-note-color!10!white, opacityback=0]

Concentration and research intensity are jointly determined: entry
expands \(N\) until operating surplus covers R\&D outlays \(x^*\).
Higher innovative potential (\(\alpha\)) supports higher concentration;
more elastic demand lowers concentration.

\end{tcolorbox}
\end{frame}

\begin{frame}{Dasgupta--Stiglitz: numerical illustration}
\protect\phantomsection\label{dasguptastiglitz-numerical-illustration}
\textbf{Using the concentration formula} with specific values:

\[
\frac{1}{N^*} = \frac{1}{\varepsilon} \cdot \frac{\alpha}{1+\alpha}
\]

\begin{itemize}
\tightlist
\item
  If \(\varepsilon = 2\) and \(\alpha = 1\):
  \(\frac{1}{N^*} = \frac{1}{2} \cdot \frac{1}{2} = \frac{1}{4}\), so
  \(N^* = 4\) firms
\item
  If \(\varepsilon = 2\) and \(\alpha = 3\):
  \(\frac{1}{N^*} = \frac{1}{2} \cdot \frac{3}{4} = \frac{3}{8}\), so
  \(N^* \approx 2.7\) firms
\end{itemize}

\begin{tcolorbox}[enhanced jigsaw, titlerule=0mm, leftrule=.75mm, toprule=.15mm, opacitybacktitle=0.6, colback=white, bottomtitle=1mm, coltitle=black, breakable, colframe=quarto-callout-note-color-frame, toptitle=1mm, title=\textcolor{quarto-callout-note-color}{\faInfo}\hspace{0.5em}{Discussion question}, arc=.35mm, rightrule=.15mm, bottomrule=.15mm, left=2mm, colbacktitle=quarto-callout-note-color!10!white, opacityback=0]

If a government subsidy increases innovative potential (\(\alpha\)),
what does the model predict for market structure? Is this
welfare-improving?

\end{tcolorbox}
\end{frame}

\begin{frame}{Dasgupta (1986): key comparative statics (I) (Dasgupta
1986)}
\protect\phantomsection\label{dasgupta-1986-key-comparative-statics-i-dasguptatheorytechnologicalcompetition1986}
\textbf{With \(P(Q) = \sigma Q^{-\varepsilon}\) and
\(c(x) = \beta x^{-\alpha}\):}

\textbf{Dynamic vs static efficiency trade-off:}

\begin{itemize}
\tightlist
\item
  \(x^*(N+1) < x^*(N)\) (more firms → less R\&D per firm)
\item
  \(Q^*(N+1) > Q^*(N)\) (more firms → higher output)
\end{itemize}

\textbf{Research intensity:}

\begin{itemize}
\tightlist
\item
  Often maximized at moderate concentration (Scherer-style inverse-U)
\end{itemize}

\textbf{Interpretation:}

\begin{itemize}
\tightlist
\item
  Market structure affects both static outcomes (\(Q\)) and dynamic
  investment incentives (\(x\))
\end{itemize}
\end{frame}

\begin{frame}{Dasgupta (1986): key comparative statics (II) (Dasgupta
1986)}
\protect\phantomsection\label{dasgupta-1986-key-comparative-statics-ii-dasguptatheorytechnologicalcompetition1986}
\textbf{Greater innovative opportunities} associated with higher
concentration:

\[
\frac{\partial (1/N^*)}{\partial \alpha} > 0
\]

\textbf{Demand growth stimulates R\&D:}

\[
\frac{\partial x^*}{\partial \sigma} > 0
\]

\textbf{Interpretation:}

\begin{itemize}
\tightlist
\item
  Larger markets (higher \(\sigma\)) increase the return to cost
  reduction
\item
  Stronger innovative opportunities (higher \(\alpha\)) can support
  higher concentration in equilibrium
\end{itemize}
\end{frame}

\begin{frame}{Summary and next week}
\protect\phantomsection\label{summary-and-next-week}
\textbf{Summary}

\begin{itemize}
\tightlist
\item
  Innovation value depends on the objective: \textbf{\(\Delta \pi\)
  (private)} versus \textbf{\(\Delta W\) (social)}
\item
  Replacement effect: pre-innovation rents reduce the incumbent's
  incremental gain from innovation
\item
  With entry threat, innovation can be worth more because it changes
  \textbf{market structure} (monopoly vs duopoly)
\item
  In oligopoly, incentives reflect competing forces (competition effect
  vs competitive advantage effect)
\end{itemize}

\textbf{Next week: patents and IPRs}

\begin{itemize}
\tightlist
\item
  Patents as incentives: monopoly rights vs.~dynamic efficiency
\item
  Patent races and timing
\item
  Disclosure, licensing, and welfare
\item
  Horizontal and vertical innovation (brief)
\end{itemize}
\end{frame}

\begin{frame}{References}
\protect\phantomsection\label{references}
\protect\phantomsection\label{refs}
\begin{CSLReferences}{1}{1}
\bibitem[\citeproctext]{ref-acemogluMarketSizeInnovation2004}
Acemoglu, Daron, and Joshua Linn. 2004. {`Market Size in Innovation:
Theory and Evidence from the Pharmaceutical Industry'}. \emph{Quarterly
Journal of Economics} 119 (3): 1049--90.
\url{https://doi.org/10.1162/0033553041502144}.

\bibitem[\citeproctext]{ref-aghionCompetitionInnovationInvertedU2005}
Aghion, Philippe, Nick Bloom, Richard Blundell, Rachel Griffith, and
Peter Howitt. 2005. {`Competition and Innovation: An Inverted-u
Relationship'}. \emph{Quarterly Journal of Economics} 120 (2): 701--28.
\url{https://doi.org/10.1093/qje/120.2.701}.

\bibitem[\citeproctext]{ref-arrowEconomicWelfareAllocation1972}
Arrow, K. J. 1972. {`Economic Welfare and the Allocation of Resources
for Invention'}. In \emph{Readings in Industrial Economics: Volume Two:
Private Enterprise and State Intervention}, edited by Charles K. Rowley.
Macmillan Education UK.
\url{https://doi.org/10.1007/978-1-349-15486-9_13}.

\bibitem[\citeproctext]{ref-blundellMarketShareMarket1999}
Blundell, Richard, Rachel Griffith, and John Van Reenen. 1999. {`Market
Share, Market Value and Innovation in a Panel of British Manufacturing
Firms'}. \emph{The Review of Economic Studies} 66 (3): 529--54.
\url{https://doi.org/10.1111/1467-937X.00097}.

\bibitem[\citeproctext]{ref-dasguptaTheoryTechnologicalCompetition1986}
Dasgupta, Partha. 1986. {`The Theory of Technological Competition'}. In
\emph{New Developments in the Analysis of Market Structure: Proceedings
of a Conference Held by the International Economic Association in
{Ottawa}, Canada}, edited by Joseph E. Stiglitz and G. Frank Mathewson.
Palgrave Macmillan UK.
\url{https://doi.org/10.1007/978-1-349-18058-5_17}.

\bibitem[\citeproctext]{ref-dasguptaIndustrialStructureNature1980}
Dasgupta, Partha, and Joseph Stiglitz. 1980. {`Industrial Structure and
the Nature of Innovative Activity'}. \emph{The Economic Journal} 90
(358): 266--93. \url{https://doi.org/10.2307/2231788}.

\bibitem[\citeproctext]{ref-gilbertPreemptivePatentingPersistence1982}
Gilbert, Richard, and David M. Newbery. 1982. {`Preemptive Patenting and
the Persistence of Monopoly'}. \emph{American Economic Review} 72 (3):
514--26.

\bibitem[\citeproctext]{ref-trajtenbergWelfareAnalysisProduct1989}
Trajtenberg, Manuel. 1989. {`The Welfare Analysis of Product
Innovations, with an Application to Computed Tomography Scanners'}.
\emph{Journal of Political Economy} 97 (2): 444--79.
\url{https://doi.org/10.1086/261611}.

\end{CSLReferences}
\end{frame}




\end{document}
